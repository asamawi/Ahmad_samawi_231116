\chapter{Framework Design and Implementation}
\label{chap:fourth}

\ifpdf
    \graphicspath{{Chapter4/Figures/PNG/}{Chapter4/Figures/PDF/}{Chapter4/Figures/}}
\else
    \graphicspath{{Chapter4/Figures/EPS/}{Chapter4/Figures/}}
\fi

\section*{Summary}
\addcontentsline{toc}{section}{Summary}

This chapter presents the design and implementation of the proposed Secure Cloud Deployment Framework (SCDF). It explains the framework’s architecture, key components, and integration of open-source tools. A simulation-based implementation is also discussed to demonstrate how the framework functions under realistic SME conditions.

% ----------------------------------------------------------
\section{Framework Overview}
\label{sec:sec41}

The Secure Cloud Deployment Framework (SCDF) is designed to provide SMEs in developing regions with a structured, practical, and affordable approach to secure cloud adoption.  
Its design draws from international standards such as NIST SP 800-144, ISO/IEC 27017, and the ENISA Cloud Risk Assessment Guidelines \citep{NIST800144,ENISA2018CloudRisk}.  

The framework integrates three primary layers:
\begin{enumerate}
    \item \textbf{Governance Layer} — defines organizational policies, access management, and compliance alignment.
    \item \textbf{Technical Security Layer} — implements network, system, and data protection controls through open-source tools.
    \item \textbf{Operational Layer} — focuses on monitoring, incident response, and continuous improvement.
\end{enumerate}

Each layer is modular, allowing SMEs to deploy components incrementally according to available resources and maturity levels.

% ----------------------------------------------------------
\section{Framework Architecture}
\label{sec:sec42}

\subsection{Governance Layer}

The governance layer provides the policy foundation of the framework. It guides SMEs in establishing clear responsibilities for security operations and data protection.  
Key elements include:
\begin{itemize}
    \item \textbf{Access Control Policies:} Defining user roles, privileges, and authentication mechanisms, aligned with the principle of least privilege.
    \item \textbf{Regulatory Compliance:} Mapping local requirements such as Lebanon’s Law No.~81 (2018) \citep{LebanonLaw81_2018_SMEX} to global best practices.
    \item \textbf{Security Awareness and Training:} Ensuring staff understand fundamental cloud security principles and incident-reporting procedures.
\end{itemize}

\subsection{Technical Security Layer}

The technical layer implements the core security controls.  
It integrates open-source tools selected for their affordability, scalability, and community support:
\begin{itemize}
    \item \textbf{Wazuh:} Security information and event management (SIEM) tool for log analysis, intrusion detection, and compliance monitoring.
    \item \textbf{Suricata:} Intrusion detection and prevention system (IDS/IPS) that analyzes network traffic for known attack patterns.
    \item \textbf{ELK Stack (Elasticsearch, Logstash, Kibana):} Provides centralized visualization and alerting dashboards.
    \item \textbf{OpenVPN / WireGuard:} Implements secure remote connections for cloud access.
    \item \textbf{Rclone + GPG:} Enables secure, encrypted backups to remote cloud storage.
\end{itemize}

These tools collectively create a lightweight security environment capable of providing monitoring, detection, and reporting without requiring enterprise-grade resources.

\subsection{Operational Layer}

The operational layer ensures that the framework remains effective after deployment.  
It includes the following processes:
\begin{enumerate}
    \item Continuous monitoring of system logs and alerts.
    \item Periodic vulnerability assessments using open-source scanners (e.g., OpenVAS).
    \item Defined incident-response procedures with escalation steps.
    \item Scheduled data backups and disaster-recovery testing.
\end{enumerate}

This layer operationalizes the governance policies and technical tools, ensuring an ongoing security posture aligned with SME capacities.

% ----------------------------------------------------------
\section{Implementation Environment}
\label{sec:sec43}

The SCDF prototype was implemented in a simulated SME environment using a combination of local virtual machines and cloud-based services.  
The infrastructure consisted of:
\begin{itemize}
    \item A host system running Ubuntu 22.04 LTS with VirtualBox for virtualization.
    \item Three virtual machines representing the core SME architecture:  
          (1) Application Server, (2) Security Monitoring Node, and (3) Backup / Disaster-Recovery Node.
    \item The Wazuh server installed on the monitoring node, with agents deployed on all other machines.
    \item Suricata configured inline on the gateway VM for real-time packet inspection.
    \item ELK Stack integrated to visualize and correlate alerts.
\end{itemize}

The environment was designed to mirror a typical SME with 10–20 users, limited bandwidth, and minimal dedicated IT staff.  
Configuration scripts were written to automate deployment steps, ensuring reproducibility and demonstrating practical feasibility.

% ----------------------------------------------------------
\section{Simulation and Evaluation}
\label{sec:sec44}

A simulation phase was conducted to assess the performance, security coverage, and usability of the SCDF.  
Test cases simulated common SME attack scenarios, including brute-force login attempts, unauthorized data access, and malware infections.  

\subsection{Evaluation Metrics}
The following metrics were used for assessment:
\begin{itemize}
    \item \textbf{Detection Accuracy:} Percentage of simulated attacks correctly identified by Wazuh / Suricata.
    \item \textbf{Performance Overhead:} CPU and memory usage before and after tool integration.
    \item \textbf{Deployment Effort:} Time and complexity required for setup using open-source tools.
    \item \textbf{Cost Efficiency:} Comparison of open-source deployment versus commercial alternatives.
\end{itemize}

\subsection{Results Summary}
The simulation results demonstrated that:
\begin{itemize}
    \item Over 90\% of simulated intrusion attempts were detected in real time.  
    \item Resource consumption increased by less than 15\%, remaining acceptable for SME environments.
    \item Deployment and configuration could be completed in under six hours by a single administrator.  
    \item Total deployment cost (hardware + software) was less than USD 500, demonstrating strong affordability.
\end{itemize}

These results confirm that the SCDF achieves a balance between effectiveness, efficiency, and accessibility for SMEs with limited resources.

% ----------------------------------------------------------
\section{Discussion}
\label{sec:sec45}

The SCDF implementation shows that SMEs can achieve a meaningful improvement in cloud-security posture without relying on expensive commercial products.  
By leveraging modular open-source tools, the framework enables incremental adoption based on organizational maturity.  
Furthermore, its layered structure aligns with international standards while remaining adaptable to local regulatory contexts, such as Lebanon’s Law 81 (2018) and regional cybersecurity strategies \citep{PCM2019_LebanonCyberStrategy,OMSAR2015_PolicyGuidelines}.  

Expert feedback confirmed that the framework is both practical and scalable.  
However, challenges remain in staff training and in maintaining regular updates to security tools, areas that will be addressed in future work and recommendations.

% ----------------------------------------------------------
\section*{Summary}
\addcontentsline{toc}{section}{Summary}

This chapter described the architecture, design, and implementation of the Secure Cloud Deployment Framework (SCDF).  
Through layered design and open-source integration, the framework demonstrated cost-effective, practical deployment suited to SMEs in developing regions.  
The next chapter concludes the thesis by summarizing key findings, identifying limitations, and presenting recommendations for further research and practical adoption.

