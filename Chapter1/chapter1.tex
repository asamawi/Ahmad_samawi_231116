% !TEX root = ../Ahmad-Samawi-231116.tex
\chapter{Background}
\label{chap:first}

\section{Introduction}

The digital transformation of small and medium-sized enterprises (SMEs) in developing regions has accelerated in recent years as organizations seek affordable ways to modernize operations and remain competitive. Cloud computing offers an attractive alternative to traditional on-premises infrastructure, providing flexibility, scalability, and cost efficiency. Yet, this transition introduces new cybersecurity risks that SMEs are often ill-equipped to manage. Limited expertise, weak governance, and resource constraints make it difficult to implement robust security measures, leaving many organizations exposed to data breaches, service disruptions, and regulatory non-compliance.  

This chapter establishes the background of the study, clarifies the motivation for the research, presents the broader context of cloud security in developing economies, and outlines the objectives that guide the proposed framework.

\section{Motivations}

Small and medium-sized enterprises (SMEs) are increasingly adopting cloud computing to reduce operational costs, improve scalability, and gain access to digital services that would otherwise require substantial upfront investment. In developing regions, cloud platforms are often viewed as an enabler of competitiveness and economic growth, allowing SMEs to overcome limitations in local infrastructure and in-house IT capabilities. As a result, cloud adoption is no longer an experimental choice for many SMEs, but a necessary step for continued operation and expansion.

Despite these advantages, the transition to cloud environments introduces significant security risks. In practice, many SMEs deploy cloud services without sufficient understanding of secure configuration, identity and access management, or shared responsibility models. Resource constraints, limited cybersecurity expertise, and an emphasis on rapid deployment frequently result in insecure default settings, weak access controls, and ad hoc security practices. These issues are not isolated incidents but recurring patterns observed across SMEs operating in resource-constrained environments.

Existing international cloud security standards and guidelines provide comprehensive recommendations for securing cloud systems. However, frameworks such as NIST SP 800-144 and ISO/IEC 27017 assume the presence of mature governance structures, specialized security roles, and the capacity to implement extensive control sets. For SMEs in developing regions, these assumptions rarely hold. The challenge is therefore not the absence of security guidance, but the lack of \textbf{practical deployment-oriented models} that translate high-level security principles into actionable steps aligned with SME capabilities.

This mismatch between comprehensive security standards and SME operational realities creates a critical gap between cloud adoption and secure cloud deployment. Addressing this gap requires a framework that emphasizes incremental implementation, affordability, and operational feasibility, enabling SMEs to improve their cloud security posture without relying on complex, resource-intensive, or proprietary solutions.

\section{Context of the Study}

The study is grounded in the specific socioeconomic and regulatory environments of developing regions. In Lebanon, the enactment of \textit{Law No.~81 (2018)} on electronic transactions and personal data represents an important legal milestone but lacks comprehensive enforcement and awareness among SMEs. Similar patterns are evident across the Middle East and Africa, where cloud adoption is growing but cybersecurity readiness remains low.  

Regional studies such as \citet{Sabbah2019_LebanonTOE} and \citet{AlRababah2023_MENA_Barriers} highlight barriers including infrastructure limitations, cost sensitivity, and insufficient training. Moreover, many SMEs rely on third-party providers without clear security agreements or incident-response capabilities, further increasing their exposure. Addressing these contextual challenges requires a framework that blends technical controls with governance, training, and policy alignment tailored to regional realities.

\section{Research Problem}

Despite the proliferation of cloud security standards and vendor-specific best practices, a persistent gap exists between available frameworks and the practical capabilities of SMEs in developing regions. The central research problem can be articulated as follows:

\begin{quote}
\textit{How can SMEs in developing regions deploy and manage cloud environments securely when existing frameworks are too complex, costly, or resource-intensive for their context?}
\end{quote}

This problem underpins the need for a structured yet lightweight Secure Cloud Deployment Framework (SCDF) that integrates affordable open-source tools, essential controls, and contextual guidance aligned with local policies.

\section{Objectives of the Study}

The overall objective of this research is to develop and validate a \textit{Secure Cloud Deployment Framework} (SCDF) for SMEs operating in developing economies.  
The specific objectives are to:
\begin{enumerate}
    \item Identify the most critical cloud-security challenges faced by SMEs in developing regions.
    \item Review and analyze existing frameworks and standards to extract adaptable controls relevant to SMEs.
    \item Design a practical, modular deployment model integrating open-source tools for monitoring, access control, and data protection.
    \item Evaluate the proposed framework through simulation and expert feedback.
\end{enumerate}

\section*{Summary}
\addcontentsline{toc}{section}{Summary}

This chapter introduced the motivation and context of the research, described the existing gap in cloud security support for SMEs in developing regions, and defined the objectives of the study. The next chapter presents a detailed review of the related literature, highlighting previous work on cloud adoption challenges, security frameworks, and SME readiness across Lebanon, the MENA region, and Sub-Saharan Africa.
