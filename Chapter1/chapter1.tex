% !TEX root = ../Ahmad-Samawi-231116.tex
\chapter{Background}
\label{chap:first}

\section{Introduction}

The digital transformation of small and medium-sized enterprises (SMEs) in developing regions has accelerated in recent years as organizations seek affordable ways to modernize operations and remain competitive. Cloud computing offers an attractive alternative to traditional on-premises infrastructure, providing flexibility, scalability, and cost efficiency. Yet, this transition introduces new cybersecurity risks that SMEs are often ill-equipped to manage. Limited expertise, weak governance, and resource constraints make it difficult to implement robust security measures, leaving many organizations exposed to data breaches, service disruptions, and regulatory non-compliance.  

This chapter establishes the background of the study, clarifies the motivation for the research, presents the broader context of cloud security in developing economies, and outlines the objectives that guide the proposed framework.

\section{Motivations}

SMEs are essential drivers of employment and innovation in developing economies but frequently lack the cybersecurity maturity of larger enterprises. Reports such as IBM’s \textit{Cost of a Data Breach 2024} indicate that misconfigurations and weak identity controls remain leading causes of cloud-related incidents. Existing international standards, including NIST~SP~800-144 and ISO/IEC~27017, offer valuable principles but are designed for large organizations with substantial budgets and specialized staff.  

For SMEs in regions such as Lebanon, North Africa, and Sub-Saharan Africa, these frameworks are often impractical to adopt in full. This reality motivates the need for a simplified, resource-efficient security model that preserves the rigor of international best practices while remaining feasible for small organizations with limited expertise.

\section{Context of the Study}

The study is grounded in the specific socioeconomic and regulatory environments of developing regions. In Lebanon, the enactment of \textit{Law No.~81 (2018)} on electronic transactions and personal data represents an important legal milestone but lacks comprehensive enforcement and awareness among SMEs. Similar patterns are evident across the Middle East and Africa, where cloud adoption is growing but cybersecurity readiness remains low.  

Regional studies such as \citet{Sabbah2019_LebanonTOE} and \citet{AlRababah2023_MENA_Barriers} highlight barriers including infrastructure limitations, cost sensitivity, and insufficient training. Moreover, many SMEs rely on third-party providers without clear security agreements or incident-response capabilities, further increasing their exposure. Addressing these contextual challenges requires a framework that blends technical controls with governance, training, and policy alignment tailored to regional realities.

\section{Research Problem}

Despite the proliferation of cloud security standards and vendor-specific best practices, a persistent gap exists between available frameworks and the practical capabilities of SMEs in developing regions. The central research problem can be articulated as follows:

\begin{quote}
\textit{How can SMEs in developing regions deploy and manage cloud environments securely when existing frameworks are too complex, costly, or resource-intensive for their context?}
\end{quote}

This problem underpins the need for a structured yet lightweight Secure Cloud Deployment Framework (SCDF) that integrates affordable open-source tools, essential controls, and contextual guidance aligned with local policies.

\section{Objectives of the Study}

The overall objective of this research is to develop and validate a \textit{Secure Cloud Deployment Framework} (SCDF) for SMEs operating in developing economies.  
The specific objectives are to:
\begin{enumerate}
    \item Identify the most critical cloud-security challenges faced by SMEs in developing regions.
    \item Review and analyze existing frameworks and standards to extract adaptable controls relevant to SMEs.
    \item Design a practical, modular deployment model integrating open-source tools for monitoring, access control, and data protection.
    \item Evaluate the proposed framework through simulation and expert feedback.
\end{enumerate}

\section*{Summary}
\addcontentsline{toc}{section}{Summary}

This chapter introduced the motivation and context of the research, described the existing gap in cloud security support for SMEs in developing regions, and defined the objectives of the study. The next chapter presents a detailed review of the related literature, highlighting previous work on cloud adoption challenges, security frameworks, and SME readiness across Lebanon, the MENA region, and Sub-Saharan Africa.
