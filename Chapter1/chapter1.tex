\chapter{Background}
\label{chap:first}

\section{Introduction}

Digital transformation has shifted from a strategic advantage to a fundamental operational necessity for small and medium enterprises (SMEs) aiming to survive in globalized markets. While early adopters sought digital tools primarily for flexibility and efficiency, today's SMEs increasingly rely on them to access advanced capabilities that are economically unviable through traditional on-premises infrastructure~\cite{Yuwono2024Information, Hari2022Cloud}. In developing regions, however, this shift is frequently driven less by long-term strategy and more by structural necessity—capital constraints, high infrastructure costs, and a scarcity of skilled technical personnel often force the hand of smaller businesses.

Cloud computing stands as the primary enabler of this transition, offering on-demand resources that effectively democratize access to enterprise-grade technology~\cite{Rani2021Cloud}. Empirical evidence from developing economies confirms growing interest among SMEs in leveraging cloud platforms for everything from core business operations to advanced analytics~\cite{Narwane2020Mediating, narwane2019factors, Owusu2020Determinants}. Yet, this migration introduces a new tier of operational risk, specifically regarding data sovereignty, service continuity, and the erosion of direct control over IT assets.

For SMEs, these risks are compounded by the complex ``shared responsibility'' models that govern cloud security. Many smaller organizations lack the governance structures required to navigate these obligations effectively, leaving them vulnerable to cyber incidents ranging from data breaches to protracted service outages~\cite{Skafi2020Factors, Tselios2022Improving}. In developing regions, the root cause of these incidents is rarely sophisticated cyber-espionage; rather, it is often simple misconfiguration, weak access controls, or a lack of basic monitoring~\cite{Pruzan2015Monitoring, JamesBalancing, Verma2024suggested, Zulkifli2023ProposedMFA}.

Critically, risk in these environments is shaped by local context. Limited technical literacy, ambiguous regulatory frameworks~\cite{Khayer2021AdoptionCC}, and intense budgetary pressure frequently result in cloud environments being deployed with insecure default settings. Studies across the Middle East, Asia, and Africa consistently highlight a recurring pattern: SMEs prioritize immediate functionality and cost reduction over security investment, deepening their exposure to cyber threats~\cite{Milhem2025integrated, Sayginer2021Multi, Owusu2020Determinants, Palanisamy2020Users, Akram2017Innovation}.

This chapter lays the foundation for the study by outlining the drivers of SME cloud adoption, the specific security challenges inherent to these environments, and the structural factors that amplify cyber risk in developing regions. These considerations underscore the urgent need for a framework that is not just secure, but practical and tailored to the unique capability constraints of SMEs.

\section{Motivations}

The primary engine driving SME cloud adoption is the potential to scale IT resources without crippling upfront capital expenditure. Cloud services allow access to sophisticated digital tools on a pay-as-you-go basis, effectively lowering the barrier to entry for digital transformation~\cite{Yuwono2024Information}. For many SMEs, this economic efficiency is the difference between modernization and obsolescence.

However, this cost-driven motivation often comes at a price. Evidence suggests that in the rush to deploy, security planning is frequently sidelined. In resource-constrained environments where speed-to-market is paramount, security is often viewed as a friction point rather than an enabler~\cite{Skafi2020Factors, Wilson2015Enablers}. Consequently, cloud environments are often established with minimal configuration hardening and informal operational practices.

This reactive approach is reinforced by a widespread ``skills gap.'' Insecure deployments often demonstrate not a willingness to accept risk, but a lack of actionable knowledge on how to mitigate it under constraint~\cite{JamesBalancing}. Security failures here are typically failures of process and expertise, not technology.

While international standards provide exhaustive lists of ``what'' should be done, they rarely explain ``how'' a resource-poor SME can achieve it. For businesses in developing regions, this creates a disconnect between high-level best practices and ground-level reality~\cite{Sayginer2021Multi}. Addressing this gap—between abstract standards and actionable implementation—is the core motivation of this research.

\section{Context of the Study}

This study is firmly situated within the socio-technical landscape of developing regions, where digital transformation is molded by specific institutional and resource constraints. SMEs in these contexts operate in regulatory environments that can be disjointed, utilize cybersecurity defenses that are often reactive, and face a persistent shortage of specialized skills.

In many developing nations, data protection laws exist on paper but lack consistent enforcement mechanisms or practical alignment with SME operations. Research from the Middle East indicates that compliance is often treated as a bureaucratic checkbox rather than a component of risk management~\cite{Milhem2025integrated, Gibreel2023Factors}. This leads to a ``compliance-security gap,'' where firms meet minimum legal standards yet remain technically vulnerable.

The Lebanese context serves as a prime example. While regulatory frameworks are present, the organizational readiness of local SMEs is often low~\cite{Skafi2020Factors}. Economic volatility and restricted investment capacity force businesses to rely on informal IT practices, limiting their ability to implement comprehensive security architectures. These challenges are not unique to Lebanon; similar patterns are observed across Africa and Asia, suggesting a systemic issue inherent to emerging digital economies~\cite{Wilson2015Enablers, Owusu2020Determinants}.

Beyond regulation, infrastructure quality remains a defining constraint. High connectivity costs and unreliable internet access often force SMEs to make trade-offs between operational continuity and security overhead~\cite{Sayginer2021Multi, Yuwono2024Information}. In this landscape, purely technical solutions that fail to account for organizational capability and local infrastructure are destined to fail.

\section{Research Problem}

Cloud computing has become a cornerstone of SME digital strategy in developing regions, yet it brings with it elevated unmanaged risk. The core problem is that SMEs operate under constraints—unreliable connectivity, scarce expertise, and tight budgets—that make standard effective security practices difficult to implement.

Existing security frameworks and standards are typically engineered for the enterprise: organizations with dedicated security operations centers (SOCs), mature governance, and deep pockets. When applied to an SME, these frameworks can appear theoretical or overwhelmingly complex. SMEs are left with a surplus of requirements but a deficit of implementation guidance.

While academic literature has extensively mapped the ``why'' of cloud adoption (intention), there is a scarcity of research addressing the ``how'' of secure deployment (operation). Moreover, existing evaluations often rely on perceptual data—what managers \textit{think} about security—rather than objective measures of security posture or operational efficiency.

This leaves a significant gap: the lack of an empirically grounded, risk-driven, and implementation-focused framework for SMEs in developing regions. Without such guidance, these businesses will continue to adopt cloud technologies in ways that are fundamentally insecure and unsustainable.

\section{Objectives of the Study}

The primary objective of this study is to design and evaluate a secure, practical, and cost-aware cloud deployment framework tailored to the needs of small and medium enterprises (SMEs) operating in developing regions. To achieve this overarching aim, the study pursues the following specific objectives:

\begin{itemize}
    \item To identify and synthesize the key security, regulatory, and operational challenges affecting cloud adoption and deployment among SMEs in developing regions, with particular attention to contextual constraints such as limited expertise, budgetary pressure, and infrastructure limitations.
    
    \item To critically analyze existing cloud security frameworks, standards, and risk-driven models in order to determine which controls and practices can be realistically adapted to SME environments.
    
    \item To design a Secure Cloud Deployment Framework (SCDF) that prioritizes essential security controls, integrates risk-based decision-making, and leverages automation and open-source tools to reduce implementation and maintenance overhead.
    
    \item To evaluate the proposed framework through simulations and illustrative case studies using measurable indicators, including security posture, cost-efficiency, and operational usability.
    
    \item To assess the practical applicability of the proposed framework in supporting secure and sustainable cloud adoption for SMEs under real-world constraints.
\end{itemize}

\section*{Summary}
\addcontentsline{toc}{section}{Summary}

This chapter established the background and context of the study by examining the drivers of cloud adoption among small and medium enterprises (SMEs) in developing regions and the security challenges that accompany this shift. It highlighted how structural constraints such as limited expertise, budgetary pressures, uneven regulatory enforcement, and infrastructure limitations contribute to elevated cyber risk in SME cloud environments.

The chapter articulated the central research problem, emphasizing the disconnect between existing cloud security frameworks—largely designed for large enterprises—and the practical realities faced by SMEs operating under resource constraints. While prior research has extensively explored factors influencing cloud adoption, comparatively limited attention has been given to deployment-oriented security frameworks that can be realistically implemented and objectively evaluated in SME contexts.

Based on this gap, the chapter defined the objectives of the study, which focus on the design and evaluation of a secure, risk-driven, and cost-aware cloud deployment framework tailored to SME needs. Together, these elements provide a structured foundation for the systematic review of related literature presented in the next chapter, which examines existing cloud security frameworks, adoption models, and relevant empirical findings in greater detail.

