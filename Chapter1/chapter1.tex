% !TEX root = ../Ahmad-Samawi-231116.tex
\chapter{Background}
\label{chap:first}

\section{Introduction}

The digital transformation of small and medium-sized enterprises (SMEs) in developing regions has accelerated in recent years as organizations seek affordable ways to modernize operations and remain competitive. Cloud computing offers an attractive alternative to traditional on-premises infrastructure, providing flexibility, scalability, and cost efficiency. Yet, this transition introduces new cybersecurity risks that SMEs are often ill-equipped to manage. Limited expertise, weak governance, and resource constraints make it difficult to implement robust security measures, leaving many organizations exposed to data breaches, service disruptions, and regulatory non-compliance.  

This chapter establishes the background of the study, clarifies the motivation for the research, presents the broader context of cloud security in developing economies, and outlines the objectives that guide the proposed framework.

\section{Motivations}

Small and medium-sized enterprises (SMEs) are increasingly adopting cloud computing to reduce operational costs, improve scalability, and gain access to digital services that would otherwise require substantial upfront investment. In developing regions, cloud platforms are often viewed as an enabler of competitiveness and economic growth, allowing SMEs to overcome limitations in local infrastructure and in-house IT capabilities. As a result, cloud adoption is no longer an experimental choice for many SMEs, but a necessary step for continued operation and expansion.

Despite these advantages, the transition to cloud environments introduces significant security risks. In practice, many SMEs deploy cloud services without sufficient understanding of secure configuration, identity and access management, or shared responsibility models. Resource constraints, limited cybersecurity expertise, and an emphasis on rapid deployment frequently result in insecure default settings, weak access controls, and ad hoc security practices. These issues are not isolated incidents but recurring patterns observed across SMEs operating in resource-constrained environments.

Existing international cloud security standards and guidelines provide comprehensive recommendations for securing cloud systems. However, frameworks such as NIST SP 800-144 and ISO/IEC 27017 assume the presence of mature governance structures, specialized security roles, and the capacity to implement extensive control sets. For SMEs in developing regions, these assumptions rarely hold. The challenge is therefore not the absence of security guidance, but the lack of \textbf{practical deployment-oriented models} that translate high-level security principles into actionable steps aligned with SME capabilities.

This mismatch between comprehensive security standards and SME operational realities creates a critical gap between cloud adoption and secure cloud deployment. Addressing this gap requires a framework that emphasizes incremental implementation, affordability, and operational feasibility, enabling SMEs to improve their cloud security posture without relying on complex, resource-intensive, or proprietary solutions.

\section{Context of the Study}

The study is grounded in the socioeconomic and regulatory contexts of developing regions where small and medium-sized enterprises (SMEs) increasingly adopt cloud computing under constrained conditions. Prior research conducted in countries such as Lebanon, Saudi Arabia, Bahrain, Ghana, and Egypt highlights that SME cloud adoption occurs within environments characterized by limited resources, varying regulatory maturity, and uneven cybersecurity preparedness.

In the Lebanese context, studies examining SME cloud adoption report the presence of regulatory and institutional constraints that influence technology implementation practices. While national legislation addressing electronic transactions and personal data protection exists, empirical research indicates that regulatory factors and organizational readiness continue to shape how SMEs approach cloud security.

Empirical studies from developing economies consistently identify barriers affecting SMEs, including infrastructure limitations, cost sensitivity, and gaps in technical skills and training. These constraints are reported across multiple country-specific investigations, suggesting that SME cloud deployment challenges are strongly influenced by local operational conditions rather than purely technical considerations.

Collectively, the reviewed literature indicates that effective cloud security approaches for SMEs in developing regions must account for both technical and organizational factors. This context underscores the need for deployment-focused frameworks that align security practices with SME capabilities and regulatory realities documented in existing empirical studies.
\section{Objectives of the Study}

The overall objective of this research is to develop and assess a deployment-oriented
\textit{Secure Cloud Deployment Framework} (SCDF) tailored to the operational
constraints of small and medium-sized enterprises (SMEs) in developing regions.

The specific objectives of the study are to:
\begin{enumerate}
    \item Identify recurring cloud security challenges reported in empirical studies of SMEs operating in developing regions.
    \item Examine existing cloud security frameworks, guidelines, and standards to identify security controls that are applicable to SME deployment contexts.
    \item Design a structured and modular cloud deployment framework that maps selected security controls to feasible implementation steps using open-source tools.
    \item Assess the feasibility and applicability of the proposed framework through controlled simulations and expert review.
\end{enumerate}

\section*{Summary}
\addcontentsline{toc}{section}{Summary}

This chapter introduced the motivation and contextual background of the research, outlined the operational challenges associated with secure cloud deployment for small and medium-sized enterprises (SMEs) in developing regions, and defined the objectives of the study. The following chapter presents a structured review of the related literature, examining prior research on cloud adoption constraints, security considerations, and SME readiness in selected developing-country contexts, including studies conducted in Lebanon and other comparable environments.

