\chapter*{Introduction}
\label{chap:introduction}
\addcontentsline{toc}{chapter}{Introduction}
\newcommand{\IntroName}{Introduction}

\section{Motivations}
\label{motivations}

Small and medium-sized enterprises (SMEs) have increasingly turned to cloud computing as a practical way to lower operational costs, scale services on demand, and gain access to digital capabilities that would otherwise be difficult to justify financially. This trend is especially visible in developing regions, where cloud platforms are often seen as a shortcut around long-standing infrastructure and capacity gaps. In theory, cloud adoption can support productivity, market access, and even broader economic growth.

In practice, however, these benefits are often accompanied by security challenges that are less obvious at the point of adoption. Many SMEs operate with limited technical expertise, tight budgets, and only a partial awareness of cyber risk. Under these conditions, cloud environments are frequently deployed with insecure default configurations, minimal access control, or informal operational practices. Compliance with data protection or privacy regulations, where such regulations exist, tends to be uneven at best.

Most widely cited cloud security frameworks—such as NIST SP~800-144 and ISO/IEC~27017—were developed with large organizations in mind. They assume structured governance, dedicated security roles, and the ability to implement extensive control sets. For SMEs in developing regions, these assumptions rarely hold. The result is not a lack of guidance, but rather guidance that is difficult to operationalize. This mismatch points to the need for a framework that is deliberately lightweight, deployment-focused, and sensitive to financial and organizational constraints, while still offering meaningful security improvement.

\section{Context of the Study}
\label{context}

This research sits at the intersection of cloud security, SME digital transformation, and cybersecurity capacity-building in developing economies. Across regions such as the Middle East, Africa, and South Asia, cloud computing has emerged as a key enabler of competitiveness for smaller firms. Yet cybersecurity maturity across these regions remains uneven, and regulatory environments are often fragmented or inconsistently enforced.

Lebanon provides a useful illustration of this dynamic. The introduction of Law No.~81 (2018) on Electronic Transactions and Personal Data represents a formal step toward regulating digital security and privacy. Still, empirical studies and industry observations suggest that awareness, enforcement, and practical guidance for SMEs remain limited. Similar conditions appear across parts of the MENA region and Sub-Saharan Africa, where small enterprises routinely face unreliable infrastructure, skills shortages, and the high cost of professional security services.

These regional realities suggest that simply importing global best practices is unlikely to be effective on its own. Instead, cloud security approaches for SMEs need to balance established standards with what is realistically achievable in local operational contexts. This study is grounded in that tension between global expectations and local feasibility.

\section{Objectives and Contributions}
\label{objectives}

The primary objective of this thesis is to design and validate a \textit{Secure Cloud Deployment Framework (SCDF)} tailored to the needs and constraints of SMEs operating in developing regions. Rather than proposing another high-level model, the framework aims to translate essential security principles into concrete, actionable deployment steps that do not require excessive cost or organizational complexity.

Specifically, this research seeks to:
\begin{itemize}
    \item Identify and analyze recurring security challenges faced by SMEs in developing regions during cloud adoption.
    \item Examine existing international standards and frameworks, with particular attention to their applicability and limitations in SME environments.
    \item Propose a modular, open-source-oriented security deployment model that integrates core cloud controls—such as identity management, monitoring, and data protection—using affordable and widely available technologies.
    \item Evaluate the feasibility of the proposed framework through simulated testing and expert review.
\end{itemize}

The expected contributions of this study include:
\begin{enumerate}
    \item A context-aware framework that helps bridge the gap between academic security models and real-world SME deployment practices.
    \item A comparative perspective on cybersecurity readiness among SMEs in Lebanon, the broader MENA region, and Sub-Saharan Africa.
    \item Practical insights and recommendations that may inform policymakers, advisors, and SME decision-makers involved in cloud security adoption.
\end{enumerate}

\section{Overview of the Thesis}
\label{overview}

This thesis is structured into five chapters. Chapter~\ref{chap:first} introduces the research problem, along with its motivation, context, and objectives. Chapter~\ref{chap:second} reviews the relevant literature on cloud computing adoption, cybersecurity challenges faced by SMEs, and existing security frameworks. Chapter~\ref{chap:third} outlines the research methodology, including data collection methods, case study selection, and the experimental setup used to evaluate the framework. Chapter~\ref{chap:fourth} presents the design and implementation of the proposed Secure Cloud Deployment Framework (SCDF), followed by evaluation results and discussion. Finally, Chapter~\ref{chap:conclusions} summarizes the main findings, discusses limitations, and suggests directions for future research.
