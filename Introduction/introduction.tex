\chapter*{Introduction}
\label{chap:introduction}
\addcontentsline{toc}{chapter}{Introduction}
\newcommand{\IntroName}{Introduction}

\section{Motivations}
\label{motivations}

Small and medium-sized enterprises (SMEs) are increasingly turning to cloud computing to reduce operational costs, enhance scalability, and access advanced digital services. This shift is particularly significant in developing regions, where cloud adoption promises to bridge technological gaps and promote economic growth. However, these benefits are often accompanied by heightened security challenges. Limited technical expertise, resource constraints, and insufficient awareness of cyber risks frequently lead to weak configurations, poor access controls, and non-compliance with data protection regulations. 

Most international cloud security frameworks, such as NIST SP 800-144 and ISO/IEC 27017, are designed for large organizations with mature IT infrastructures and dedicated security teams. SMEs in developing regions rarely have the financial or human capital to implement such comprehensive standards. This gap creates an urgent need for a practical, lightweight, and cost-effective framework that enables SMEs to deploy cloud services securely and sustainably within their contextual limitations.

\section{Context of the Study}
\label{context}

This research is situated at the intersection of cloud security, SME digital transformation, and cybersecurity capacity-building in developing economies. In regions such as the Middle East, Africa, and South Asia, cloud computing is emerging as a key enabler of innovation and competitiveness. Yet, cybersecurity maturity remains low, and the regulatory landscape is fragmented. 

In Lebanon, for example, the adoption of Law No.\ 81 (2018) on Electronic Transactions and Personal Data marks a step toward formalizing digital security and privacy principles, but practical enforcement and awareness among SMEs remain limited. Similar patterns are observed across the broader MENA region and Sub-Saharan Africa, where studies show that small enterprises face significant challenges related to infrastructure reliability, skills shortages, and the affordability of professional security services. These regional realities highlight the need for context-aware solutions that balance global best practices with local feasibility.

\section{Objectives and Contributions}
\label{objectives}

The main objective of this thesis is to design and validate a \textit{Secure Cloud Deployment Framework (SCDF)} tailored to the needs and constraints of SMEs operating in developing regions. The framework aims to provide a clear set of guidelines and tools that enable small organizations to adopt secure cloud infrastructures without excessive cost or complexity.

The specific goals of the research are as follows:
\begin{itemize}
    \item Identify and analyze the key security challenges faced by SMEs in developing regions during cloud adoption.
    \item Review existing international standards and frameworks, highlighting their strengths and limitations in SME contexts.
    \item Propose a modular, open-source-based security deployment model that integrates essential cloud controls (such as access management, monitoring, and data protection) using affordable technologies.
    \item Validate the proposed framework through simulated testing and expert evaluation.
\end{itemize}

The expected contributions of this study include:
\begin{enumerate}
    \item A context-sensitive framework that bridges the gap between academic research and SME implementation practices.
    \item A comparative analysis of regional cybersecurity readiness for SMEs in Lebanon, the MENA region, and Sub-Saharan Africa.
    \item Practical recommendations for policymakers and SME decision-makers seeking to improve cloud security adoption.
\end{enumerate}

\section{Overview of the Thesis}
\label{overview}

This thesis is organized into five chapters. Chapter \ref{chap:first} introduces the research problem, its motivation, context, and objectives. Chapter \ref{chap:second} presents a comprehensive review of the literature on cloud computing adoption, SME cybersecurity challenges, and existing security frameworks. Chapter \ref{chap:third} details the research methodology, including data collection, case study selection, and experimental setup for framework validation. Chapter \ref{chap:fourth} describes the design and implementation of the proposed Secure Cloud Deployment Framework (SCDF), along with evaluation results and discussion. Finally, Chapter \ref{chap:conclusions} concludes the thesis by summarizing the key findings, limitations, and recommendations for future research.

---

