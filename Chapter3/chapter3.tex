\chapter{Research Methodology}
\label{chap:third}

\ifpdf
    \graphicspath{{Chapter3/Figures/PNG/}{Chapter3/Figures/PDF/}{Chapter3/Figures/}}
\else
    \graphicspath{{Chapter3/Figures/EPS/}{Chapter3/Figures/}}
\fi

\section*{Summary}
\addcontentsline{toc}{section}{Summary}

This chapter presents the methodology adopted to design, develop, and validate the proposed Secure Cloud Deployment Framework (SCDF). It describes the overall research design, data collection approach, framework development process, and validation techniques. The goal of the methodology is to ensure that the proposed framework is both theoretically grounded and practically applicable within the context of SMEs operating in developing regions.

% ----------------------------------------------------------
\section{Research Design}
\label{sec:sec31}

\begin{itemize}
    \item \textbf{Type:} Mixed-methods research (Qualitative + Quantitative).
    \item \textbf{Phases:}
    \begin{enumerate}
        \item \textbf{Exploratory Review:} Literature search on SME challenges and framework gaps.
        \item \textbf{Framework Design:} Integrating theory (TOE) with technical standards (NIST/ISO) into SCDF.
        \item \textbf{Validation (Sim):} Implementation in controlled cloud env with open-source tools.
        \item \textbf{Validation (Expert):} Assessment by subject-matter experts for usability/completeness.
    \end{enumerate}
    \item \textbf{Goal:} Ensure conceptual soundness + Operational feasibility.
\end{itemize}

% ----------------------------------------------------------
\section{Research Approach}
\label{sec:sec32}

\begin{itemize}
    \item \textbf{Methodology:} Design Science Research (DSR).
    \item \textbf{Artifact:} Secure Cloud Deployment Framework (SCDF).
    \item \textbf{DSR Cycle:}
    \begin{itemize}
        \item Problem Identification (SME constraints).
        \item Design \& Development (Modular framework).
        \item Demonstration (Simulated deployment).
        \item Evaluation (Expert feedback).
        \item Communication (Thesis/Publications).
    \end{itemize}
\end{itemize}

% ----------------------------------------------------------
\section{Data Collection and Sources}
\label{sec:sec33}

\begin{itemize}
    \item \textbf{Secondary Data:} Academic papers, Technical reports (ENISA), Policy docs (Lebanon Law 81).
    \item \textbf{Primary Data:} Expert interviews + Experimental results.
\end{itemize}

\subsection{Secondary Data}

\begin{itemize}
    \item Analysis of regional policies (MENA/SSA) and standards.
    \item Identification of recurring adoption barriers.
\end{itemize}

\subsection{Primary Data}

\begin{itemize}
    \item \textbf{Method:} Semi-structured interviews + Experiments.
    \item \textbf{Focus:}
    \begin{enumerate}
        \item Core security challenges.
        \item Feasible open-source tools.
        \item Organizational/Regulatory constraints.
    \end{enumerate}
\end{itemize}

% ----------------------------------------------------------
\section{Framework Development Process}
\label{sec:sec34}

\begin{enumerate}
    \item \textbf{Requirement ID:} Mapping SME needs vs. Standards (NIST, ISO) \& Local Policy.
    \item \textbf{Architecture:} Governance + Technical + Operational layers.
    \item \textbf{Integration:} Selection of tools (Wazuh, Suricata, ELK).
    \item \textbf{Validation Prep:} Definition of metrics (Time, Resource, Coverage).
\end{enumerate}

% ----------------------------------------------------------
\section{Validation and Evaluation}
\label{sec:sec35}

\subsection{Simulation-Based Validation}

\begin{itemize}
    \item \textbf{Setup:} Virtualized SME environment.
    \item \textbf{Tools:} Wazuh, Suricata, Open Source configs.
    \item \textbf{Metrics:}
    \begin{itemize}
        \item Security Coverage (Identity, Network, Data).
        \item Performance Overhead (CPU/RAM).
        \item Ease of Deployment (Time).
    \end{itemize}
\end{itemize}

\subsection{Expert Evaluation}

\begin{itemize}
    \item \textbf{Reviewers:} Academic/Industry experts.
    \item \textbf{Criteria:} Relevance, Completeness, Usability, Scalability.
\end{itemize}

% ----------------------------------------------------------
\section*{Summary}
\addcontentsline{toc}{section}{Summary}

\begin{itemize}
    \item Chapter details: DSR methodology, Data collection, Development steps, Validation.
    \item Next layout: Implementation results and Framework description.
\end{itemize}

