\chapter{Secure Cloud Deployment Framework (SCDF)}
\label{chap:third}

\ifpdf
    \graphicspath{{Chapter3/Figures/PNG/}{Chapter3/Figures/PDF/}{Chapter3/Figures/}}
\else
    \graphicspath{{Chapter3/Figures/EPS/}{Chapter3/Figures/}}
\fi

% ----------------------------------------------------------
\section{Chapter Overview and Positioning}
\label{sec:chapter3_overview}

This chapter presents the Secure Cloud Deployment Framework (SCDF)---a structured, evidence-driven approach tailored for Small and Medium Enterprises (SMEs) to deploy cloud infrastructure securely, with emphasis on operational feasibility under real-world constraints.

Cloud adoption offers SMEs scalability, cost efficiency, and flexibility; however, security risks and skill gaps remain critical concerns that commonly hinder secure implementation and long-term operations. According to cybersecurity analyses focused on SMEs, fundamental challenges such as limited budgets, scarcity of specialized expertise, and low security awareness significantly increase vulnerability to threats and misconfigurations in cloud environments. This is affirmed by the European Union Agency for Cybersecurity (ENISA), which notes that SMEs frequently lack clear, tailored guidance suited to their unique operational context, and that such guidance is essential for safe cloud adoption and risk management.

SCDF arises from a juxtaposition of two realities:

\begin{itemize}
    \item \textbf{Operational necessity:} Cloud platforms are increasingly essential for SME digital transformation, supporting everything from basic infrastructure to business continuity.
    
    \item \textbf{Security challenge:} SMEs often operate without formal cybersecurity governance, dedicated teams, or enterprise-grade tools, yet face the same spectrum of threats that larger organizations do---including misconfigurations, identity compromise, and ransomware exposure.
\end{itemize}

This chapter translates systematic insights from the literature into SCDF---a framework that operationalizes key security principles into actionable and verifiable actions suitable for SMEs on Infrastructure-as-a-Service (IaaS) platforms. Rather than aiming for comprehensive enterprise compliance (e.g., ISO/IEC 27001 certification), SCDF focuses on deployable minimum viable security mechanisms that measurably reduce risk while preserving accessibility and cost-effectiveness.

% ----------------------------------------------------------
\section{Scope Definition and Applicability Boundaries}
\label{sec:scope_definition}

This section defines the scope of SCDF's applicability and delineates the boundaries of what the framework does and does not address. These definitions ensure that SCDF's recommendations remain realistic, actionable, and relevant for the intended user group---Small and Medium Enterprises (SMEs) operating in cloud environments with distinct constraints.

\subsection{Target SME Profile}
\label{subsec:target_sme_profile}

SCDF is intended primarily for SMEs that share a set of organizational, technical, and economic characteristics common in developing and resource-constrained environments. SME research highlights that such organizations often adopt cloud computing to overcome infrastructural limitations and cost barriers, but concurrently struggle with security and privacy concerns, limited technical experience, and minimal governance structures. These challenges contribute to hesitancy in cloud adoption despite strong perceived benefits such as scalability and cost efficiency.

An SME appropriate for SCDF typically exhibits the following features:

\begin{itemize}
    \item \textbf{Organizational scale:} Fewer than 250 employees with limited or no dedicated cybersecurity personnel, making specialized security teams impractical.
    
    \item \textbf{Technical capacity:} Reliance on generalist IT staff, outsourced IT support, or ad-hoc technical management without specialized cybersecurity training. This lack of in-house expertise is widely recognized as a critical barrier to secure cloud adoption in SMEs.
    
    \item \textbf{Cloud consumption model:} Primary use of Infrastructure-as-a-Service (IaaS) offerings, including virtual machines, storage, and basic networking, rather than complex platform or container ecosystems.
    
    \item \textbf{Economic constraints:} High cost sensitivity and preference for cost-effective or open-source tooling, as is common among SMEs evaluating cloud security tools.
    
    \item \textbf{Operational environment:} Unreliable connectivity and limited administrative bandwidth, common in developing regions where cloud adoption is essential yet operational challenges remain.
\end{itemize}

This SME profile reflects a modal class of organizations that significantly benefit from cloud computing but face constraints that make conventional enterprise-scale security frameworks impractical.

\subsection{Explicit Exclusions}
\label{subsec:explicit_exclusions}

To maintain realism and utility, SCDF explicitly excludes the following objectives and control categories. These exclusions are not oversights but intentional boundaries grounded in normative research on SMEs' limited capacity for exhaustive security frameworks and compliance activities:

\begin{itemize}
    \item \textbf{Formal certification pursuit:} SCDF does not provide the comprehensive control documentation or audit structures required to earn certifications such as ISO/IEC 27001, SOC 2, or equivalent compliance attestations. While such standards offer valuable guidance, they often require significant governance resources that exceed the operational capacity of target SMEs. SME guidance specifically emphasizes that organizations cannot always implement full enterprise-grade frameworks due to time, expertise, and financial constraints.
    
    \item \textbf{Enterprise-scale Zero Trust architectures:} Full Zero Trust implementations---involving pervasive identity attestation, micro-segmentation, and continuous contextual authorization---demand advanced tooling and governance structures typically outside SME capacity at initial phases. SCDF instead incorporates risk-appropriate identity and access principles that embody Zero Trust philosophies without architectural complexity.
    
    \item \textbf{Dedicated Security Operations Center (SOC) capabilities:} Round-the-clock threat hunting, specialist incident response teams, and continuous human monitoring exceed the operational budget and personnel availability of most SMEs. SCDF emphasizes automated detection with prioritized alerts to balance visibility with resource expense.
    
    \item \textbf{Provider obligation substitution:} SCDF recognizes but does not replace the cloud provider's responsibilities for physical infrastructure, hypervisor integrity, and connectivity backbone security. The framework operates strictly within the shared responsibility model and leaves provider-side assurances to established cloud contractual and architectural practices.
\end{itemize}

By clearly defining what SCDF does not aim to achieve, the framework maintains focus, feasibility, and applicability for its intended users without overpromising outcomes inconsistent with SME realities.

% ----------------------------------------------------------
\section{Derivation of Framework Requirements from SLR Findings}
\label{sec:derivation_requirements}

The design of the Secure Cloud Deployment Framework (SCDF) is systematically derived from evidence synthesized in the Systematic Literature Review (SLR) presented in Chapter 2. This derivation translates recurring themes in cloud security literature --- particularly those relating to SME constraints and practical threat landscapes --- into actionable framework requirements that address real-world risk vectors. Contemporary systematic reviews of cloud security identify a range of threats and mitigation strategies that are directly relevant to SMEs. These include identity and access vulnerabilities, misconfigurations, DDoS and malware threats, data breaches, and the need for security awareness and vulnerability management techniques such as IAM, SIEM, and encryption \citep{Milhem2025integrated,Al2023Security}. Importantly, multiple reviews also highlight that cloud misconfigurations --- incorrect permissions, weak IAM settings, overly permissive networks, and reduced monitoring --- are consistently among the top causes of cloud security incidents and exploited vulnerabilities \citep{Gibreel2023Factors,Sabah2024Cloud}.

The SCDF requirements are mapped to these empirical findings through a structured goal-to-requirement transformation, as shown in Table~\ref{tab:slr_to_requirements}. Each requirement targets measurable improvement in areas identified repeatedly across SLR outputs.

\begin{table}[htbp]
    \centering
    \caption{Mapping SLR Findings to SCDF Requirements}
    \label{tab:slr_to_requirements}
    \resizebox{\textwidth}{!}{%
    \begin{tabular}{|p{5cm}|p{4cm}|p{5cm}|}
        \hline
        \textbf{SLR Finding} & \textbf{Identified Constraint} & \textbf{SCDF Requirement} \\
        \hline
        Misconfiguration is a prominent contributor to cloud security issues (permissions, network ACLs, APIs) \citep{Gibreel2023Factors,Chen2023Securing}. & Limited SME expertise and human errors & Secure-by-default baselines and automated configuration validation procedures \\
        \hline
        Identity and access weaknesses (account hijacking, weak credentials, privilege misuse) \citep{Al2023Security,Trigueros2024Security}. & Lack of trained security personnel and complex IAM & Constrained guided workflows for IAM and enforced multi-factor identity controls \\
        \hline
        High threat diversity --- from malware to data breaches to denial of service attacks \citep{Tanimu2023Review,Gupta2023Deep}. & Insufficient threat visibility and response capability & Incremental monitoring controls with verified alerting thresholds and prioritized response \\
        \hline
        Cost and complexity inhibit adoption of enterprise-grade security tools \citep{Skafi2020Factors,Wilson2015Enablers,Sayginer2021Multi}. & Severe budget limitations and tooling barriers & Explicit preference for OSS, freemium tools, and cloud-native controls with trade-off documentation \\
        \hline
        High variability in cloud deployments (providers, architectures, workloads) \citep{Rashid2024Analysis,Abdulsalam2023Security}. & Low operational applicability of generic frameworks & Concrete control mapping with provider-agnostic and provider-specific implementation notes and step-by-step workflows \\
        \hline
        SLA, governance, and formal risk management are often absent in SMEs \citep{Chan2023Digital,Matias2019Cloud}. & Absence of formal governance structures & Lightweight governance templates with accountability roles and risk acceptance logs \\
        \hline
    \end{tabular}%
    }
\end{table}

\subsection{Validation of Requirements}

Each derived requirement addresses a documented failure mode from the literature. For example, the requirement for secure-by-default baselines responds directly to findings that SMEs frequently deploy with default configurations due to lack of implementation guidance \citep{Milhem2025integrated,Gibreel2023Factors}. Similarly, the preference for open-source tooling addresses the economic constraints identified across multiple regional studies \citep{Skafi2020Factors,Wilson2015Enablers,Sayginer2021Multi}.

% ----------------------------------------------------------
\section{Design Principles and Trade-Off Rationale}
\label{sec:design_principles}

This section establishes the foundational design principles that govern the Secure Cloud Deployment Framework (SCDF) and explains the explicit trade-offs made to balance security effectiveness with deployability for Small and Medium Enterprises (SMEs). Each principle is grounded in current cloud security understanding and reflects patterns observed in cloud security literature and practical SME guidance.

Cloud security frameworks universally emphasize the need for risk prioritization, control consistency, and measurable outcomes \citep{Chen2023Securing,Al2023Security}. For SMEs with limited security expertise and resources, SCDF's design principles adapt these general concepts into a pragmatic, incremental, and outcome-oriented approach that avoids over-prescription and unnecessary complexity. This aligns with guidance from SME-focused cloud security frameworks, which recommend simplified risk assessment and control implementation tailored to organizational capacity rather than enterprise-grade compliance mandates \citep{Skafi2020Factors,Wilson2015Enablers}.

\subsection{Core Design Principles}
\label{subsec:core_principles}

Each SCDF principle reflects a design trade-off between ideal security constructs and the practical realities SMEs face. These principles are intended to guide both what SMEs should implement and how they should approach security progression in cloud environments.

\begin{enumerate}
    \item \textbf{Risk-Driven Prioritization}
    
    In cloud environments, certain controls offer disproportionately higher security impact relative to effort --- especially in SME contexts where attack vectors such as identity compromise or misconfigurations are common. For this reason, SCDF orders controls by risk frequency and impact, emphasizing identity, network, and configuration hardening before more advanced areas.
    
    This principle aligns with risk-focused guidance in cloud security research that highlights prioritized mitigation of identity and access threats and misconfiguration avoidance \citep{Gibreel2023Factors,Trigueros2024Security}. It also resonates with the Identify and Protect components of established cybersecurity models (e.g., NIST CSF), which prioritize risk assessment and protective measures early in a security lifecycle.
    
    \item \textbf{Minimal Viable Security}
    
    SCDF aims for minimum viable security, defined as the smallest set of effective controls that reduce critical risks to an acceptable level. Rather than seeking exhaustive control coverage, this principle favors practical sufficiency --- implementing just enough security that significantly reduces risk without overwhelming limited operational capacity.
    
    SME cloud security literature consistently observes that over-complex frameworks discourage adoption, especially when SMEs lack security staff \citep{Chan2023Digital,Matias2019Cloud}. This principle echoes the Secure by Design approach, where security is built into system configurations early and in a targeted manner rather than patched in later reactively \citep{Rashid2024Analysis}.
    
    \item \textbf{Automation as Compensation, Not Replacement}
    
    Automated tooling (e.g., configuration scanners, cloud posture management) helps address skill gaps by enforcing consistency and reducing drift, but it does not eliminate the need for human oversight. SCDF leverages automation to enforce repeatable validation and routine tasks (e.g., baseline assessment), yet explicitly retains human judgment for strategic decisions and incident responses.
    
    Modern cloud monitoring paradigms and continuous threat exposure approaches emphasize that automation improves detection and response efficiency but must be paired with human risk decisions \citep{Tanimu2023Review,Gupta2023Deep}.
    
    \item \textbf{Incremental Maturity}
    
    Reflecting the incremental nature of security maturity, SCDF defines four deployment phases that build sequentially. This allows SMEs to achieve incremental risk reduction while learning and allocating resources gradually.
    
    Incremental maturity mirrors continuous monitoring and maturity models in cloud security best practices, proposing progressive improvements rather than all-at-once adoption \citep{Abdulsalam2023Security,Sabah2024Cloud}.
    
    \item \textbf{Provider Agnosticism with Pragmatic Exceptions}
    
    SCDF's controls are designed to be implementable across multiple IaaS platforms, but vendor-specific native controls are permitted when they reduce cost and complexity without compromising security outcomes. This reflects the practical reality that SMEs often prefer cloud-native services for integrated functionality and minimal operational overhead.
    
    SME guidance acknowledges that cloud provider native tools can be leveraged effectively to meet baseline security needs with lower cost and integration effort \citep{Sayginer2021Multi,Milhem2025integrated}.
\end{enumerate}

\subsection{Explicit Trade-Offs}
\label{subsec:explicit_tradeoffs}

Deploying security within resource-constrained organizations requires explicit acknowledgement of trade-offs. The following articulates the key tensions SCDF resolves:

\textbf{Breadth vs Deployability}

SCDF deliberately limits coverage to the most impactful controls initially, postponing niche or advanced capabilities to later phases. The rationale is that controls which remain unimplemented due to complexity offer no protection at all; partial implementation of priority controls enhances risk reduction meaningfully. This trade-off parallels cloud security best-practice guidance that promotes prioritization of high-impact controls (e.g., IAM, logging) before expanding to less impactful domains \citep{Chen2023Securing,Rashid2024Analysis}.

\textbf{Advanced vs Baseline Protection}

Sophisticated detection and analytics (e.g., behavior-based analytics, custom detection rules) are valuable, but are deferred until baseline security, monitoring, and hardening are stable. SCDF prioritizes foundational protections (MFA, least-privilege access, default-deny networking) before introducing complex monitoring that demands more expertise. Cloud monitoring best practices emphasize foundational logging and SIEM readiness before layering advanced analytics for threat hunting \citep{Tanimu2023Review,Gupta2023Deep}.

\textbf{Vendor Neutrality vs Pragmatic Efficiency}

While the framework promotes core principles that are conceptual and provider-agnostic, it endorses the use of vendor-native tools where they are demonstrably easier to implement and maintain. This reflects guidance advocating the pragmatic use of cloud-native security capabilities alongside open-source tooling for SMEs \citep{Sayginer2021Multi,Skafi2020Factors}.

\subsection{Summary of Design Principles and Trade-Offs}
\label{subsec:design_summary}

The design of SCDF explicitly balances effectiveness and practicality. It embeds core best practices from cloud security literature into a framework that assumes limited resources, limited expertise, and the need for measurable outcomes. The principles are not theoretical ideals but a composition of documented patterns of risk and mitigation strategies validated in cloud security guidance and empirical studies \citep{Al2023Security,Gibreel2023Factors,Trigueros2024Security}.

% ----------------------------------------------------------
\section{Conceptual Structure of the SCDF}
\label{sec:conceptual_structure}

The Secure Cloud Deployment Framework (SCDF) organizes its guidance into a three-layer architecture that aligns with established information security principles --- separating governance, control implementation, and operations. This layered approach mirrors the defense-in-depth architecture common in cloud and IT security, where multiple, complementary layers of protection are applied to mitigate threats and reduce attack surface \citep{Chen2023Securing,Al2023Security}.

A layered structure helps SMEs reason about who decides (governance), what is implemented (controls), and how it runs daily (operations), without conflating strategic oversight with tactical execution.

\subsection{Framework Layers}
\label{subsec:framework_layers}

SCDF is structured into the following three core layers:

\begin{enumerate}
    \item \textbf{Governance Layer}
    
    The Governance Layer establishes policies, roles, and accountability mechanisms that guide all other security activities. It answers strategic questions such as:
    \begin{itemize}
        \item What security objectives align with the SME's business needs?
        \item Who is responsible for approving risk decisions?
        \item What policies govern acceptable use, exceptions, and escalation paths?
    \end{itemize}
    
    This layer is essential for clarifying decision authority, documenting risk acceptance, and ensuring that operational activities are anchored in organizational intent rather than ad-hoc technician decisions. In established cloud and cybersecurity frameworks, governance is recognized as foundational and prerequisite to effective control implementation --- guiding risk appraisal and policy enforcement \citep{Gibreel2023Factors,Trigueros2024Security}.
    
    \item \textbf{Control Layer}
    
    The Control Layer translates governance directives into technical and procedural controls that protect the SME's cloud environment. It includes:
    \begin{itemize}
        \item \textbf{Identity and Access Management (IAM):} Restricting access via secure authentication and least-privilege policies, a control repeatedly identified as foundational in cloud security literature \citep{Tanimu2023Review}.
        \item \textbf{Network Security:} Firewalls, security groups, and segmentation concepts (including defense-in-depth patterns) to limit unauthorized access \citep{Rashid2024Analysis}.
        \item \textbf{Data Security:} Use of encryption and access policies to protect sensitive assets.
        \item \textbf{Monitoring and Logging:} Mechanisms for detecting anomalous behaviors, configuration changes, and potential compromise.
    \end{itemize}
    
    According to cloud security control taxonomies, such layered controls are essential to protect data, compute, and network assets from evolving threats and maintain confidentiality, integrity, and availability \citep{Abdulsalam2023Security,Sabah2024Cloud}.
    
    The Control Layer embodies the what --- the set of safeguards that enforce governance policies in the cloud environment.
    
    \item \textbf{Operational Layer}
    
    The Operational Layer defines executable workflows, automation scripts, and response procedures that translate controls into daily practice. This includes:
    \begin{itemize}
        \item Routine security configuration validation
        \item Scheduled automated scans
        \item Alert triage and escalation playbooks
        \item Backup execution and recovery rehearsals
    \end{itemize}
    
    The Operational Layer represents how controls are executed on a recurring basis, enabling SMEs to operationalize security through repeatable and measurable actions. Cloud security best practices emphasize that controls without operational execution --- such as regular scans, alerts, or backups --- achieve little. Embedding detection and response in day-to-day practice is essential for actual risk reduction rather than theoretical compliance \citep{Gupta2023Deep,Milhem2025integrated}.
\end{enumerate}

\subsubsection{Layer Interaction Model}
\label{subsubsec:layer_interaction}

The three layers interact through top-down direction and bottom-up feedback:
\begin{itemize}
    \item \textbf{Governance informs Controls:} Governance policies and risk decisions define which controls must be implemented and enforced.
    \item \textbf{Controls shape Operations:} Controls define the operational activities required; operations execute and validate these controls.
    \item \textbf{Operations feed into Governance:} Operational outcomes (e.g., metrics, incident reports) inform governance reviews and policy updates, ensuring continuous improvement.
\end{itemize}

This feedback loop echoes iterative cybersecurity lifecycle recommendations found in frameworks like continuous threat exposure management (CTEM), where ongoing validation and adaptation are key to maintaining resilience \citep{Skafi2020Factors,Wilson2015Enablers}.

\subsection{Deployment Phases}
\label{subsec:deployment_phases}

Implementation follows four sequential phases, each building on the previous, as shown in Figure~\ref{fig:deployment_phases}.

\begin{figure}[htbp]
    \centering
    \begin{tikzpicture}[
        phase/.style={rectangle, rounded corners, minimum width=3cm, minimum height=1.5cm, text centered, align=center, font=\small\bfseries, draw, thick},
        arrow/.style={->, >=stealth, very thick},
        focus/.style={font=\scriptsize, align=center, text width=3cm}
    ]
        % Phase boxes
        \node[phase, fill=blue!15] (p1) at (0,0) {Phase 1\\Baseline};
        \node[phase, fill=green!15] (p2) at (4,0) {Phase 2\\Hardening};
        \node[phase, fill=orange!15] (p3) at (8,0) {Phase 3\\Monitoring};
        \node[phase, fill=red!15] (p4) at (12,0) {Phase 4\\Recovery};
        
        % Arrows
        \draw[arrow] (p1) -- (p2);
        \draw[arrow] (p2) -- (p3);
        \draw[arrow] (p3) -- (p4);
        
        % Focus areas
        \node[focus, below=0.5cm of p1] {Identity\\Network\\Logging};
        \node[focus, below=0.5cm of p2] {Config Standards\\Vulnerability Mgmt\\Access Control};
        \node[focus, below=0.5cm of p3] {SIEM\\Alerting\\Audit};
        \node[focus, below=0.5cm of p4] {Backups\\Recovery Plans\\Continuity};
        
        % Risk reduction indicator
        \node[above=1cm of p2, font=\small, align=center] {\textbf{Risk Reduction Priority}\\(Highest \textrightarrow Lower)};
    \end{tikzpicture}
    \caption{SCDF Four-Phase Deployment Sequence}
    \label{fig:deployment_phases}
\end{figure}

Implementation follows four sequential phases, each building on the previous:

\begin{enumerate}
    \item \textbf{Phase 1: Baseline Establishment} --- Secure the foundation including identity management, network boundaries, and initial logging.
    
    \item \textbf{Phase 2: Risk Hardening} --- Implement configuration standards, vulnerability management, and access controls.
    
    \item \textbf{Phase 3: Monitoring and Detection} --- Deploy continuous monitoring, alerting, and audit capabilities.
    
    \item \textbf{Phase 4: Backup, Recovery, and Continuity} --- Ensure resilience through automated backups and documented recovery procedures.
\end{enumerate}

These phases are ordered by risk reduction per unit effort. Phase 1 controls (identity and network) address the highest-impact threat vectors and therefore take priority over Phase 4 controls (backup and recovery), despite the importance of both.

\subsection{Phase Advancement Criteria}
\label{subsec:phase_advancement_criteria}

To ensure objective and measurable progression through the SCDF deployment phases, SCDF adopts a \textbf{hybrid advancement model} (Option C) that combines minimum tool completion requirements with quantitative risk metric thresholds. This approach mitigates the limitations of checklist-only methods---which risk superficial compliance---and metric-only methods---which risk advancement without foundational controls present.

This hybrid model reflects best practices in security assessment frameworks that utilize Goal-Question-Metrics (GQM) approaches and risk-based evaluation to quantitatively assess security posture and maturity levels tailored to SMEs. By requiring both control presence and demonstrated risk reduction, SCDF ensures that advancement represents genuine security improvement rather than administrative compliance.

Each advancement step requires:
\begin{enumerate}
    \item Completion of essential controls and tooling appropriate for that phase, and
    \item Achievement of quantitative risk reduction or maturity thresholds as measured by accepted security assessment tools and methodologies.
\end{enumerate}

\subsubsection{Phase 1 $\rightarrow$ Phase 2 Advancement Criteria}
\label{subsubsec:p1_to_p2}

An SME may advance to Phase 2 (Risk Hardening) only when:

\begin{enumerate}
    \item \textbf{Mandatory Controls Are Implemented}
    \begin{itemize}
        \item Identity and Access Management (IAM) policies configured per least-privilege principles.
        \item Multi-Factor Authentication (MFA) enabled for all privileged accounts and administrative access.
        \item Network security groups or equivalent (e.g., firewall rules) deployed and attached to all compute instances.
    \end{itemize}
    
    \item \textbf{Quantitative Evidence of Configuration Risk Reduction}
    \begin{itemize}
        \item A cloud posture assessment (e.g., Prowler, ScoutSuite, or equivalent) shows \textbf{no critical findings} in identity and network configuration categories within the last 7 days.
        \item This aligns with industry emphasis on strong access control and secure configuration as foundational cloud security controls in SME-targeted risk guidance.
    \end{itemize}
    
    \item \textbf{Verification}
    \begin{itemize}
        \item A dated automated assessment report must be provided demonstrating the configuration posture.
        \item SME management must formally acknowledge residual risk and its acceptance.
    \end{itemize}
\end{enumerate}

\subsubsection{Phase 2 $\rightarrow$ Phase 3 Advancement Criteria}
\label{subsubsec:p2_to_p3}

Advancement to Phase 3 (Monitoring and Detection) requires:

\begin{enumerate}
    \item \textbf{Tools Deployed}
    \begin{itemize}
        \item Vulnerability scanning tools operational (e.g., Nessus Essentials, OpenVAS).
        \item Host-level hardening checks performed using industry benchmarks (e.g., CIS Benchmarks).
        \item Automated patch management configured where supported.
    \end{itemize}
    
    \item \textbf{Quantitative Thresholds}
    \begin{itemize}
        \item Weekly vulnerability scans must show \textbf{no high-severity unpatched vulnerabilities older than 30 days}.
        \item Automated configuration drift detection shows \textbf{\textless 5\% deviation} from established secure baselines.
        \item These thresholds operationalize cloud security maturity objectives using repeated measurements as indicators of ongoing risk management.
    \end{itemize}
    
    \item \textbf{Verification}
    \begin{itemize}
        \item Two consecutive weekly scan reports meeting thresholds.
        \item Documented exception handling for any residual high-severity findings with business justification.
    \end{itemize}
\end{enumerate}

\subsubsection{Phase 3 $\rightarrow$ Phase 4 Advancement Criteria}
\label{subsubsec:p3_to_p4}

Advancement to Phase 4 (Recovery and Continuity) requires:

\begin{enumerate}
    \item \textbf{Tools in Place}
    \begin{itemize}
        \item Security Information and Event Management (SIEM) or centralized log aggregation and alerting systems operational with documented alert rules for critical events.
        \item Defined automated response playbooks for prioritized alert categories.
    \end{itemize}
    
    \item \textbf{Quantitative Monitoring Metrics}
    \begin{itemize}
        \item Simulated test incidents executed: Mean Time to Detection (MTTD) \textbf{\textless 24 hours} for predefined critical events.
        \item Measured alert signal-to-noise ratio \textbf{\textgreater 70\%} for critical event types (i.e., $\leq$ 30\% false alarms), indicating meaningful detection precision rather than overwhelming noise.
        \item This emphasis on validated detection performance reflects emerging continuous threat exposure management principles that prioritize measurable detection capability rather than raw log volume.
    \end{itemize}
    
    \item \textbf{Verification}
    \begin{itemize}
        \item Documented test report showing incident detection times and alert quality metrics.
        \item Demonstrated SME operator proficiency in triage and escalation procedures.
    \end{itemize}
\end{enumerate}

\subsubsection{Rationale for the Hybrid Model}
\label{subsubsec:hybrid_rationale}

The hybrid advancement model addresses two key pitfalls:

\begin{itemize}
    \item \textbf{Binary-only advancement} risks ``checkbox security''---where controls exist but are ineffective due to misconfiguration or improper use (e.g., MFA enabled but with SMS-based factors vulnerable to SIM-swapping, or security groups present but allowing 0.0.0.0/0 access).
    
    \item \textbf{Quantitative-only advancement} risks progression based on fortunate scan results rather than the presence of foundational controls or secure baseline adherence (e.g., no vulnerabilities detected simply because the scanning tool is not configured to assess the actual attack surface).
\end{itemize}

By requiring both control presence and evidence of security improvement, SCDF bridges practical implementation with measurable security posture enhancement. This balanced approach is consistent with cloud security guidance tailored for resource-constrained environments, which emphasizes practical risk metrics over rigid compliance checklists.

SMEs may remain in any given phase until criteria are met; no artificial time limits are imposed that could otherwise compromise real security gains.

% ----------------------------------------------------------
\section{Operational Workflow and Responsibility Model}
\label{sec:operational_workflow}

\subsection{Actor Roles}
\label{subsec:actor_roles}

SCDF recognizes three primary actors in SME cloud security:

\begin{itemize}
    \item \textbf{SME Owner / Management:} Holds ultimate accountability for risk acceptance and business prioritization. Decides security investment levels and accepts residual risk where full mitigation is infeasible.
    
    \item \textbf{Technical Operator:} Implements and maintains controls. May be internal IT staff, external managed service provider, or part-time administrator. Responsible for execution, not strategic risk decisions.
    
    \item \textbf{Automated Systems:} Enforce configuration, conduct monitoring, generate alerts, and execute approved automated responses (e.g., IP blocking, backup initiation).
\end{itemize}

This separation clarifies that while technical operators implement controls, risk decisions remain with accountable management.

\subsection{Decision versus Automation Boundary}
\label{subsec:decision_boundary}

SCDF distinguishes between activities suitable for automation and those requiring human judgment:

\begin{itemize}
    \item \textbf{Automated (no human latency):} Configuration validation, scheduled backups, routine log aggregation, basic pattern-based alerting.
    
    \item \textbf{Human-driven (requires judgment):} Access approval for privileged accounts, incident escalation after initial detection, recovery decision-making during outages, policy exceptions.
\end{itemize}

This boundary ensures that automation reduces routine burden without displacing essential human oversight where judgment is required.

% ----------------------------------------------------------
\section{Tooling and Control Mapping}
\label{sec:tooling_mapping}

\subsection{Tool Selection Criteria}
\label{subsec:tool_selection_criteria}

All tool recommendations in SCDF satisfy the following criteria, as illustrated in Figure~\ref{fig:tool_taxonomy}:

\begin{figure}[htbp]
    \centering
    \begin{tikzpicture}[
        box/.style={rectangle, rounded corners, minimum width=2.5cm, minimum height=0.8cm, text centered, font=\scriptsize, draw, thick},
        criterion/.style={ellipse, minimum width=3cm, minimum height=1cm, text centered, align=center, font=\small\bfseries, fill=blue!10, draw},
        arrow/.style={->, >=stealth}
    ]
        % Central SME node
        \node[criterion, fill=blue!20] (sme) at (0,0) {SME\\Constraints};

        % Four criteria
        \node[criterion] (cost) at (-4,2) {Cost Access};
        \node[criterion] (complex) at (4,2) {Complexity};
        \node[criterion] (maint) at (-4,-2) {Maintenance};
        \node[criterion] (compat) at (4,-2) {Compatibility};

        % Tool examples
        \node[box, fill=green!10] at (-4,3.2) {Free/OSS/Freemium};
        \node[box, fill=green!10] at (4,3.2) {No Security Expertise};
        \node[box, fill=green!10] at (-4,-3.2) {Active Community};
        \node[box, fill=green!10] at (4,-3.2) {Multi-Cloud};

        % Arrows
        \draw[arrow] (sme) -- (cost);
        \draw[arrow] (sme) -- (complex);
        \draw[arrow] (sme) -- (maint);
        \draw[arrow] (sme) -- (compat);
    \end{tikzpicture}
    \caption{SCDF Tool Selection Criteria}
    \label{fig:tool_taxonomy}
\end{figure}

\begin{enumerate}
    \item \textbf{Cost accessibility:} Free, open-source, or freemium with sufficient capability for SME-scale deployments.
    \item \textbf{Complexity appropriateness:} Deployable by generalist IT staff without specialized security expertise.
    \item \textbf{Maintenance sustainability:} Active project with community or vendor support; not deprecated or abandonware.
    \item \textbf{Provider compatibility:} Functional across multiple IaaS platforms or clearly scoped to specific compatible platforms.
\end{enumerate}

\subsection{Identity and Access Management (IAM)}
\label{subsec:iam_tools}

Table~\ref{tab:iam_tools} lists IAM tooling options categorized by deployment phase and implementation effort.

\begin{table}[htbp]
    \centering
    \caption{SCDF IAM Tooling Options by Phase}
    \label{tab:iam_tools}
    \resizebox{\textwidth}{!}{%
    \begin{tabular}{|l|p{5cm}|l|l|l|}
        \hline
        \textbf{Tool/Service} & \textbf{Purpose} & \textbf{Phase} & \textbf{Effort} & \textbf{Cost} \\
        \hline
        Native Cloud IAM & Provider-managed identity (AWS IAM, Azure AD, GCP IAM, OCI IAM, etc.) & P1 & Low & Included \\
        \hline
        MFA (Native/TOTP) & Multi-factor authentication using authenticator apps & P1 & Low & Free \\
        \hline
        Keycloak & Self-hosted SSO and identity federation & P2 & Medium & OSS \\
        \hline
        Auth0 & SaaS authentication and authorization platform & P2 & Low & Freemium \\
        \hline
        Authelia & Lightweight SSO for self-hosted apps & P3 & Medium & OSS \\
        \hline
    \end{tabular}%
    }
\end{table}

\textbf{Critical Control:} Phase 1 must implement MFA for all privileged accounts before any additional hardening.

\subsection{Network and Edge Protection}
\label{subsec:network_tools}

Table~\ref{tab:network_tools} presents network protection tooling organized by deployment phase.

\begin{table}[htbp]
    \centering
    \caption{SCDF Network Protection Tooling by Phase}
    \label{tab:network_tools}
    \resizebox{\textwidth}{!}{%
    \begin{tabular}{|l|p{5cm}|l|l|l|}
        \hline
        \textbf{Tool/Service} & \textbf{Purpose} & \textbf{Phase} & \textbf{Effort} & \textbf{Cost} \\
        \hline
        Security Groups & Instance-level and subnet-level network access control & P1 & Low & Included \\
        \hline
        Network ACLs & Subnet-level stateless traffic filtering & P1 & Low & Included \\
        \hline
        Cloudflare (Free) & Edge WAF, DDoS protection, DNS security & P2 & Low & Freemium \\
        \hline
        Native Provider WAF & Application-layer attack filtering & P2 & Medium & Usage-based \\
        \hline
        OpenSnitch/Portmaster & Host-level application firewall & P3 & Medium & OSS \\
        \hline
        WireGuard/Tailscale & Modern VPN for secure access & P2 & Low & OSS \\
        \hline
    \end{tabular}%
    }
\end{table}

\textbf{Critical Control:} Default-deny security groups must be established in Phase 1, with explicit allow rules only for required services.

\subsection{Configuration Hardening and Audit}
\label{subsec:hardening_tools}

Table~\ref{tab:hardening_tools} lists configuration assessment tools mapped to deployment phases. These tools address the misconfiguration risk identified as the dominant cause of cloud security incidents in the SLR.

\begin{table}[htbp]
    \centering
    \caption{SCDF Configuration Hardening Tooling by Phase}
    \label{tab:hardening_tools}
    \resizebox{\textwidth}{!}{%
    \begin{tabular}{|l|p{4.5cm}|l|l|p{3cm}|}
        \hline
        \textbf{Tool/Service} & \textbf{Purpose} & \textbf{Phase} & \textbf{Effort} & \textbf{Coverage} \\
        \hline
        Prowler & Multi-cloud security posture assessment & P2 & Medium & AWS, Azure, GCP \\
        \hline
        ScoutSuite & Multi-cloud security auditing & P2 & Medium & AWS, Azure, GCP, Alibaba \\
        \hline
        CloudSploit & Continuous cloud security monitoring & P3 & Low & Multi-cloud \\
        \hline
        CIS Benchmarks & Industry-standard configuration guidance & P2 & High & All major IaaS \\
        \hline
        CIS-CAT Lite & Automated configuration assessment & P3 & Medium & Standardized \\
        \hline
        Lynis & Host-level security auditing & P2 & Low & Linux/Unix systems \\
        \hline
        OpenSCAP & Automated compliance checking & P3 & Medium & SCAP-compliant systems \\
        \hline
    \end{tabular}%
    }
\end{table}

\textbf{Automation Priority:} Weekly automated Prowler/ScoutSuite scans should be scheduled in Phase 3, with findings routed to designated response personnel.

\subsection{Monitoring and Logging}
\label{subsec:monitoring_tools}

Table~\ref{tab:monitoring_tools} presents monitoring and logging tooling critical for detecting the anomalous activity and misconfigurations that SLR findings identified as common initial attack vectors.

\begin{table}[htbp]
    \centering
    \caption{SCDF Monitoring and Logging Tooling by Phase}
    \label{tab:monitoring_tools}
    \resizebox{\textwidth}{!}{%
    \begin{tabular}{|l|p{4.5cm}|l|l|p{3cm}|}
        \hline
        \textbf{Tool/Service} & \textbf{Purpose} & \textbf{Phase} & \textbf{Effort} & \textbf{Notes} \\
        \hline
        Cloud-native logging & Provider-managed log collection & P1 & Low & Basic P1 requirement \\
        \hline
        Wazuh & Open-source SIEM and HIDS & P3 & High & Single-node deployment \\
        \hline
        Graylog & Log aggregation and analysis & P3 & Medium & Simpler than ELK \\
        \hline
        OpenSearch & Search and analytics suite & P3 & High & ELK alternative \\
        \hline
        Loki & Lightweight log aggregation & P3 & Low & Grafana ecosystem \\
        \hline
        Prometheus & Monitoring and alerting & P3 & Medium & Metrics collection \\
        \hline
        Grafana & Visualization and dashboards & P3 & Low & OSS, widely used \\
        \hline
        Uptime Kuma & Simple uptime monitoring & P2 & Low & Quick deployment \\
        \hline
    \end{tabular}%
    }
\end{table}

\textbf{SLR-Based Rationale:} Wazuh addresses the monitoring gaps identified in Lebanese and Jordanian SME studies by providing centralized visibility without requiring specialized SOC personnel \citep{Skafi2020Factors,Al2018Factors}.

\subsection{Backup and Recovery}
\label{subsec:backup_tools}

Table~\ref{tab:backup_tools} lists backup and recovery tooling essential for business continuity---addressing the connectivity instability and disaster exposure identified in Philippines and Indonesia SLR contexts \citep{Matias2019Cloud,Chan2023Digital}.

\begin{table}[htbp]
    \centering
    \caption{SCDF Backup and Recovery Tooling by Phase}
    \label{tab:backup_tools}
    \resizebox{\textwidth}{!}{%
    \begin{tabular}{|l|p{4.5cm}|l|l|p{3cm}|}
        \hline
        \textbf{Tool/Service} & \textbf{Purpose} & \textbf{Phase} & \textbf{Effort} & \textbf{Encryption} \\
        \hline
        Cloud snapshots & VM and volume-level backups & P1 & Low & Provider-managed \\
        \hline
        Restic & Encrypted, deduplicated backups & P4 & Medium & AES-256 \\
        \hline
        Duplicati & Encrypted file-level backup & P4 & Low & AES-256 \\
        \hline
        BorgBackup & Deduplicating archiver & P4 & Medium & AES-256 \\
        \hline
        Kopia & Fast encrypted backups & P4 & Low & AES-256 \\
        \hline
        rclone & Cloud sync and backup & P4 & Low & Optional \\
        \hline
        Object Storage & Off-site backup destination & P4 & Low & Client-side \\
        \hline
    \end{tabular}%
    }
\end{table}

\textbf{Resilience Requirement:} Phase 4 must implement automated daily backups with 30-day retention, tested monthly via restoration exercises.

\subsection{Vulnerability Management}
\label{subsec:vulnerability_tools}

Table~\ref{tab:vulnerability_tools} presents vulnerability management tooling addressing the patching and exposure management challenges identified across multiple regional SLR studies.

\begin{table}[htbp]
    \centering
    \caption{SCDF Vulnerability Management Tooling by Phase}
    \label{tab:vulnerability_tools}
    \resizebox{\textwidth}{!}{%
    \begin{tabular}{|l|p{4.5cm}|l|l|p{3cm}|}
        \hline
        \textbf{Tool/Service} & \textbf{Purpose} & \textbf{Phase} & \textbf{Effort} & \textbf{Coverage} \\
        \hline
        Nessus Essentials & Vulnerability scanning (up to 16 IPs) & P2 & Low & Network, host \\
        \hline
        OpenVAS/Greenbone & Comprehensive VA solution & P2 & High & Network, web apps \\
        \hline
        Trivy & Container and IaC scanning & P2 & Low & Containers, OS \\
        \hline
        Grype & Container vulnerability scanner & P2 & Low & Container images \\
        \hline
        Snyk (Free) & OSS dependency scanning & P2 & Low & Code libraries \\
        \hline
        Clair & Container static analysis & P2 & Medium & Container registries \\
        \hline
        Unattended-upgrades & Automatic OS patching & P2 & Low & Debian/Ubuntu \\
        \hline
        dnf-automatic & Automatic OS patching & P2 & Low & RHEL/CentOS/Fedora \\
        \hline
    \end{tabular}%
    }
\end{table}

\textbf{SLR-Based Requirement:} Weekly automated scanning (Trivy for containers, Nessus for hosts) addresses the systematic patching gaps identified in Lebanese and Turkish SME contexts \citep{Skafi2020Factors,Sayginer2021Multi}.

\subsection{Master Tool-to-Phase Mapping Summary}
\label{subsec:tool_mapping_summary}

Table~\ref{tab:master_tool_mapping} provides a consolidated summary of all SCDF-recommended tools organized by deployment phase. This mapping enables SMEs to plan phased implementation according to immediate risk reduction priorities.

\begin{table}[htbp]
    \centering
    \caption{SCDF Master Tool-to-Phase Mapping}
    \label{tab:master_tool_mapping}
    \resizebox{\textwidth}{!}{%
    \begin{tabular}{|l|p{6cm}|l|l|}
        \hline
        \textbf{Phase} & \textbf{Tools/Controls} & \textbf{Core Risk Addressed} & \textbf{Cumulative Effort} \\
        \hline
        \textbf{P1: Baseline} & Security Groups, Cloud IAM, MFA, Basic Logging, Snapshots & Initial access via identity or network & Low (1-2 days) \\
        \hline
        \textbf{P2: Hardening} & WAF, VPN, Prowler, Lynis, Nessus, Trivy, Auth0/Keycloak & Misconfiguration and lateral movement & Medium (+3-5 days) \\
        \hline
        \textbf{P3: Monitoring} & Wazuh, Graylog, Grafana, CloudSploit, CIS-CAT Lite & Unmonitored breaches & High (+5-10 days) \\
        \hline
        \textbf{P4: Recovery} & Restic, Duplicati, Kopia, DR procedures & Data loss and service continuity & Medium (+2-3 days) \\
        \hline
    \end{tabular}%
    }
\end{table}

\textbf{Implementation Guidance:} The cumulative effort estimates assume a single part-time technical operator with generalist IT skills. Organizations with external managed service providers may achieve faster deployment by outsourcing Phase 3 implementation while maintaining internal responsibility for governance decisions.

% ----------------------------------------------------------
\section{IaaS Provider Coverage}
\label{sec:provider_coverage}

\subsection{Dominant Cloud Infrastructure Providers}
\label{subsec:dominant_providers}

SCDF targets the following dominant IaaS providers that collectively represent approximately 80\% of global cloud infrastructure spending:

\begin{itemize}
    \item Amazon Web Services (AWS)
    \item Microsoft Azure
    \item Google Cloud Platform (GCP)
    \item Alibaba Cloud
    \item Oracle Cloud Infrastructure (OCI)
    \item IBM Cloud
    \item Tencent Cloud
\end{itemize}

This coverage ensures applicability to organizations using major hyperscale providers. The framework's provider-agnostic core controls are portable across these platforms, with provider-specific implementation notes where necessary.

\subsection{SME-Focused and Cost-Efficient IaaS Providers}
\label{subsec:sme_providers}

While representing a smaller share of global infrastructure spend, the following providers are operationally significant for SMEs due to pricing simplicity, predictable costs, and regional availability:

\begin{itemize}
    \item Hetzner Cloud
    \item DigitalOcean
    \item OVHcloud
    \item Linode (Akamai)
    \item Vultr
    \item Scaleway
\end{itemize}

SCDF is designed to function effectively across both hyperscale and SME-focused providers, recognizing that cost-sensitive organizations may prioritize these alternatives while still requiring robust security controls.

% ----------------------------------------------------------
\section{Evaluation Strategy}
\label{sec:evaluation_strategy}

SCDF effectiveness is evaluated through three objective dimensions:

\begin{enumerate}
    \item \textbf{Security Posture Indicators:}
    \begin{itemize}
        \item Reduction in critical misconfigurations vs. default deployments
        \item Control coverage percentage (controls implemented vs. recommended)
        \item Time-to-detection for simulated incidents
    \end{itemize}
    
    \item \textbf{Operational Overhead:}
    \begin{itemize}
        \item Total deployment time for each phase
        \item Number of manual steps versus automated steps
        \item Staff time required for ongoing maintenance
    \end{itemize}
    
    \item \textbf{Cost Efficiency:}
    \begin{itemize}
        \item Direct costs (tooling, infrastructure)
        \item Indirect cost proxies (automation reducing manual effort)
        \item Cost comparison vs. equivalent commercial solutions
    \end{itemize}
\end{enumerate}

These metrics are assessed through simulation-based validation and expert evaluation to ensure that SCDF delivers measurable security improvement without prohibitive operational burden.

% ----------------------------------------------------------
\section{Chapter Summary}
\label{sec:chapter3_summary}

This chapter established the Secure Cloud Deployment Framework (SCDF) as a practical, evidence-driven response to the security challenges facing SMEs in developing regions. Key elements defined include:

\begin{itemize}
    \item \textbf{Scope boundaries} that ensure realism through explicit exclusions and a clear target SME profile.
    \item \textbf{Systematic derivation from SLR findings}, mapping documented constraints to actionable framework requirements.
    \item \textbf{Design principles} that prioritise risk-driven, incremental, and automation-assisted security.
    \item \textbf{Layered framework structure} separating governance, controls, and operations into manageable components.
    \item \textbf{Four-phase deployment sequence} ordered by risk reduction per unit effort.
    \item \textbf{Tooling catalogue} mapped to security domains, with all tools satisfying cost and complexity constraints for SMEs.
    \item \textbf{Provider coverage} spanning both dominant hyperscale and SME-focused IaaS platforms.
\end{itemize}

The framework design balances security effectiveness with deployability, recognizing that a framework unimplemented provides no protection regardless of its theoretical completeness. The following chapter validates SCDF through practical implementation scenarios and objective measurement against the evaluation criteria established herein.
