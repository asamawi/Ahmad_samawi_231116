\chapter{Literature Review}
\label{chap:second}

\ifpdf
    \graphicspath{{Chapter2/Figures/PNG/}{Chapter2/Figures/PDF/}{Chapter2/Figures/}}
\else
    \graphicspath{{Chapter2/Figures/EPS/}{Chapter2/Figures/}}
\fi

\section{Introduction}
\label{sec:sec21}

Cloud computing is increasingly championed as a great equalizer for small and medium-sized enterprises (SMEs), promising access to enterprise-grade resources that were once the exclusive domain of large multinational corporations~\cite{Misra2022Factors, Yuwono2024Information, Rani2021Cloud}. In developing regions, this narrative is particularly potent: cloud adoption is often framed not just as an upgrade, but as a mechanism to leapfrog structural and economic barriers. However, the reality on the ground is more complex. Evidence suggests that for many SMEs, the journey to the cloud remains uneven, fraught with unmanaged risks, and frequently stalled by local constraints.

Existing literature indicates that cloud adoption among SMEs cannot be understood solely through technological or economic advantages. Instead, adoption decisions are shaped by a complex interaction of organizational readiness, cybersecurity risk perception, regulatory uncertainty, operational capacity, and local contextual constraints. SMEs in developing regions frequently operate in environments characterized by unreliable connectivity, limited technical expertise, constrained budgets, and fragmented regulatory frameworks. These conditions do not merely delay cloud adoption; they fundamentally alter the risk profile associated with cloud deployment and amplify the consequences of security failures.

Although numerous studies have examined cloud adoption drivers and barriers, much of the existing research treats cybersecurity as one factor among many, rather than as a structural condition that shapes deployment feasibility and sustainability. Furthermore, many widely cited cloud security frameworks and standards were designed for large enterprises with mature governance structures, dedicated security teams, and stable regulatory environments. When applied to SMEs in developing regions, these frameworks often prove overly complex, resource-intensive, or misaligned with local realities.

This chapter presents a critical review of the literature on cloud computing adoption and cybersecurity challenges facing SMEs in developing regions. Rather than offering a descriptive summary of prior work, the chapter synthesizes empirical findings across security, regulatory, operational, and contextual dimensions to identify persistent limitations in existing models and frameworks. Particular attention is given to how local constraints—such as connectivity limitations, skills shortages, and financial restrictions—function as risk amplifiers rather than isolated barriers.

By critically examining adoption theories, security frameworks, practical tools, and evaluation approaches, this chapter establishes the conceptual and empirical foundations for the development of a Secure Cloud Deployment Framework (SCDF) tailored to SME contexts. The synthesis provided herein directly informs the research problem and motivates the methodological and design choices presented in subsequent chapters.


% ----------------------------------------------------------
\section{Research Questions}
\label{sec:research_questions}

In light of the identified gaps and the need for a targeted approach to SME cloud security in developing regions, this study seeks to answer the following research questions:

\begin{enumerate}
    \item What security, regulatory, and operational challenges hinder cloud adoption among SMEs in developing regions, and how do local constraints such as connectivity, limited expertise, and budget influence their risk exposure?
    \item What existing cloud security frameworks, risk-driven models, and automation tools can be adapted to develop a secure, cost-effective, and maintainable cloud deployment framework for SMEs?
    \item How can the proposed secure cloud deployment framework be evaluated and validated in SME contexts through simulations, pilot deployments, or case studies using measurable metrics such as security posture, cost-efficiency, and usability?
\end{enumerate}

% ----------------------------------------------------------
\section{Research Methodology}
\label{sec:methodology}

This thesis employs a systematic literature review methodology to ensure a transparent, unbiased, and replicable research process. The review protocol was defined prior to the execution of the study and includes the research questions, search strategy, inclusion and exclusion criteria, quality assessment criteria, and data extraction plan.

Relevant studies were identified through comprehensive searches of major scientific databases, including ACM Digital Library (\url{http://portal.acm.org}), Google Scholar (\url{https://scholar.google.com/}), IEEE Digital Library (\url{http://ieeexplore.ieee.org}), ScienceDirect (\url{http://www.sciencedirect.com}), Scopus (\url{http://www.scopus.com}), and Springer Link (\url{http://link.springer.com}).

\subsection{Search Strategy}

A comprehensive search strategy was developed to identify relevant literature related to cloud security frameworks, adoption models, and cybersecurity practices pertinent to small and medium-sized enterprises (SMEs) in developing regions. The strategy combined domain-specific keywords with methodological terms using Boolean operators. Search strings were adapted to the syntax of each selected database.

The search focused on peer-reviewed journal articles and conference papers published from 2015 onwards. The generic search string structure is presented below:

\begin{verbatim}
TITLE-ABS-KEY(
    ("cloud computing" OR "cloud adoption" OR "cloud migration") 
    AND 
    ("SME*" OR "small and medium enterprise*" OR "small business*") 
    AND 
    ("developing countr*" OR "emerging econom*" OR "Global South" 
     OR "low-income countr*") 
    AND 
    (security OR privacy OR regulation OR compliance OR risk 
     OR challenge OR barrier OR cost OR connectivity 
     OR expertise OR automation OR framework)
)
AND PUBYEAR > 2014
\end{verbatim}

\subsubsection*{Search Adaptation}
To ensure comprehensive coverage, the search strategy included the following adaptations:
\begin{itemize}
    \item Keywords were adjusted based on specific database requirements and syntax.
    \item Truncation and wildcard symbols were used where supported to capture variations in terminology.
    \item Search fields included title, abstract, and keywords to maximize the retrieval of relevant studies.
\end{itemize}

\subsection{Inclusion and Exclusion Criteria}

To ensure that the review focused on high-quality and relevant studies, specific inclusion and exclusion criteria were established. These criteria were applied during both the initial screening and the full-text review phases.

\subsubsection*{Inclusion Criteria}
Studies were included if they met the following methodological and thematic requirements:
\begin{itemize}
    \item Focus on SMEs or similar small organizations.
    \item Related to cloud security, risk, adoption challenges, or frameworks.
    \item Coverage of technical security mechanisms, frameworks, implementation guidelines, or regional adoption.
    \item Empirical studies, systematic reviews, framework proposals, and case studies.
    \item Propose or evaluate tools, frameworks, or methods.
    \item Peer-reviewed journal or conference papers.
    \item Studies published between 2015--2025.
\end{itemize}

\subsubsection*{Exclusion Criteria}
Studies were excluded if they fell into any of the following categories:
\begin{itemize}
    \item Duplicate publications.
    \item Focus only on large enterprises.
    \item Unrelated to cloud computing or security.
    \item Purely theoretical papers without practical implications.
    \item Non-Empirical and Opinion-Based Publications.
    \item Secondary and Tertiary Studies.
    \item Student theses, posters, or editorials.
    \item Papers without access to full text or sufficient abstracts.
    \item Not available in English.
\end{itemize}

\subsection{Study Selection Process}

The study selection process was conducted in accordance with the PRISMA (Preferred Reporting Items for Systematic Reviews and Meta-Analyses) guidelines. The PRISMA flowchart illustrates the progression of studies through the different stages of the review.

An initial search across the selected databases yielded 75 records. After the removal of duplicates, 66 unique studies remained. During the title and abstract screening phase, 3 studies were excluded due to irrelevance to the research scope. The full texts of 32 articles were then assessed for eligibility, resulting in 28 studies included in the final analysis.

The PRISMA flowchart provides a clear and structured summary of this process, supporting the transparency and reproducibility of the review.

\begin{figure}[htbp]
    \centering
    \includegraphics[width=\textwidth]{Figures/PRISMA_flow.png}
    \caption{PRISMA flowchart of the study selection process.}
    \label{fig:prisma_flow}
\end{figure}

\subsection{Quality Assessment of Selected Studies}
To ensure the reliability and scientific rigor of the selected literature, a quality assessment (QA) was conducted as part of the systematic literature review process. The purpose of this assessment was to evaluate the methodological soundness, relevance, and clarity of each study before its inclusion in the final analysis.

Each selected study was assessed using a predefined set of quality criteria. The criteria were designed to reflect best practices in empirical research within computer science and engineering disciplines. For each criterion, a score was assigned based on the degree to which the study satisfied the requirement.

\begin{table}[htbp]
    \centering
    \caption{Quality Assessment Criteria}
    \label{tab:qa_criteria}
    \begin{tabular}{|l|p{4cm}|p{8cm}|}
        \hline
        \textbf{ID} & \textbf{Quality Criterion} & \textbf{Description} \\
        \hline
        QA1 & SME Focus & Does this study focus on small and medium enterprises (SMEs) rather than exclusively on large enterprises or multinational corporations? \\
        \hline
        QA2 & Challenge Domain Coverage & Does this study report on security, regulatory, or operational aspects of cloud computing? \\
        \hline
        QA3 & Cloud Adoption Relevance & Does this study examine cloud adoption challenges, barriers, or cloud security frameworks/solutions? \\
        \hline
        QA4 & Empirical Evidence & Is this an empirical study (including case studies, pilot deployments, simulations, surveys, interviews, or systematic reviews) rather than an opinion piece, editorial, or purely theoretical paper without practical applications? \\
        \hline
        QA5 & Geographic Scope & Is this study conducted in or focusing on developing regions/countries rather than exclusively in developed countries with mature IT infrastructure? \\
        \hline
    \end{tabular}
\end{table}

Each criterion was scored as follows:
\begin{itemize}
    \item 1: Criterion fully satisfied (yes)
    \item 0.5: Criterion partially satisfied (maybe)
    \item 0: Criterion not satisfied (no)
\end{itemize}
The total quality score for each study was calculated by summing the individual scores. Studies scoring below 2.5 were excluded from the final synthesis. This process helped ensure that only high-quality and relevant studies contributed to the review findings.

For each included study, data were systematically extracted and organized. The extracted data included information on: Study Context (geographic region/country studied, industry sector(s) of SMEs examined, SME size characteristics, development level of region, study methodology, time period of data collection); Cloud Adoption Barriers (security challenges, regulatory barriers, operational challenges, technical barriers, specific examples and percentages); Local Constraints (connectivity issues, limited expertise, budget constraints, other regional factors, how these constraints specifically increase risk exposure or adoption barriers); Frameworks and Solutions (name and type of framework, key features and capabilities, specifically designed for or adaptable to SMEs, automation capabilities mentioned, cost-effectiveness considerations, maintainability aspects, adaptations); Evaluation Methods (evaluation methodology, specific metrics used for assessment, validation approaches and their effectiveness, sample sizes and study duration when applicable); and Key Findings (main conclusions, effectiveness of proposed solutions or frameworks, quantitative results, qualitative insights about SME-specific needs, success factors and best practices identified, gaps, future research recommendations).

A qualitative synthesis was then conducted to identify patterns, trends, and gaps across the literature.

\section{Cloud Computing Adoption among SMEs}
\label{sec:sec23}

Cloud computing adoption among small and medium-sized enterprises (SMEs) has been extensively examined in the literature, often framed as a rational response to cost pressures, scalability requirements, and competitive dynamics. In principle, cloud services enable SMEs to access enterprise-grade infrastructure and applications without substantial upfront investment, thereby lowering barriers to digital transformation. However, evidence from developing regions indicates that adoption decisions are rarely driven by technological advantages alone.

Research consistently points to cost efficiency, scalability, and flexibility as the primary drivers of SME cloud adoption. The shift from capital-intensive infrastructure to operational expenditure models allows resource-constrained firms to punch above their weight, aligning IT consumption directly with business fluctuations. For these businesses, the cloud is often seen less as a technology and more as a survival tool for market responsiveness.

Despite these perceived benefits, adoption rates and usage maturity remain uneven across developing regions. Survey-based studies from the Middle East, South Asia, and Sub-Saharan Africa demonstrate that positive perceptions of cloud computing do not necessarily translate into adoption decisions or secure deployment. In several contexts, perceived usefulness and performance benefits were found to have weak or statistically insignificant influence on adoption intention, challenging assumptions embedded in dominant technology acceptance models.

Instead, adoption behavior among SMEs in developing regions appears to be shaped primarily by organizational readiness and environmental conditions. Top management support repeatedly emerges as a decisive factor, influencing both initial adoption and sustained use. SMEs with prior IT experience and exposure to digital systems demonstrate greater willingness to migrate to cloud platforms, while firms with limited internal capabilities exhibit risk-averse behavior even when potential benefits are acknowledged.

Environmental uncertainty further complicates adoption decisions. Unreliable infrastructure, volatile economic conditions, and ambiguous regulatory environments introduce risks that are external to the cloud technology itself but directly affect its perceived feasibility. In such settings, cloud adoption is evaluated not as a purely efficiency-enhancing innovation, but as a strategic risk decision. SMEs operating in unstable environments demonstrate heightened sensitivity to potential service disruptions, data loss, and vendor dependency, often delaying adoption until external conditions improve.

The literature also highlights sectoral and contextual variation in adoption patterns. SMEs engaged in e-commerce or digitally native activities tend to adopt cloud services earlier due to operational necessity, whereas firms in traditional sectors exhibit greater resistance. However, even among early adopters, cloud usage is frequently limited to basic services, with advanced functionalities underutilized due to skills shortages and security concerns.

Overall, the reviewed studies suggest that cloud computing adoption among SMEs in developing regions cannot be adequately explained by benefit-centric or technology-centric models alone. Adoption decisions are embedded within a broader socio-technical context where organizational capacity, managerial perception of risk, and environmental constraints play a dominant role. For instance, in the context of Indian SMEs, Narwane et al.~\cite{narwane2019factors} utilized Structural Equation Modelling (SEM) to demonstrate that ``sharing and collaboration'' capabilities were among the strongest positive drivers for adoption, ranking alongside cost savings. These findings imply that secure cloud deployment for SMEs requires frameworks that go beyond adoption intention and explicitly address readiness, risk management, and post-adoption sustainability rather than assuming linear progression from perceived benefit to effective use~\cite{Skafi2020Factors, Narwane2020Mediating, Owusu2020Determinants}.

% ----------------------------------------------------------
\section{Cloud Adoption Barriers Facing SMEs}
\label{sec:sec24}

The reviewed literature identifies a set of interrelated barriers that collectively hinder cloud adoption among SMEs in developing regions, particularly within the Middle East and emerging Asian economies. These barriers extend beyond purely technical considerations and span security, regulatory, operational, and infrastructural domains. Rather than acting independently, they reinforce one another and significantly shape both adoption decisions and post-adoption risk exposure.

\subsection{Security Challenges}

Security concerns emerge as the most consistently reported barrier across the reviewed studies. Data protection and privacy risks were identified in nearly all empirical investigations~\cite{Palanisamy2020Users} examining cloud adoption among SMEs. Perceived loss of control over data hosted in external environments remains a central concern, particularly in contexts where trust in service providers and regulatory safeguards is limited.

Empirical evidence illustrates substantial variation in the salience of security concerns across regions. In Jordan, approximately 14.3\% of SME managers expressed reluctance to adopt cloud computing primarily due to security considerations, while a significant majority (71.4\%) actually believed that large cloud providers offered better security mechanisms than they could manage internally~\cite{Al2018Factors}. This contrasts sharply with findings from Lebanon, where 56\% of SMEs cited security and compliance as major barriers~\cite{Skafi2020Factors}. This disparity reflects differences in environmental stability and market maturity rather than purely technical awareness.

Specific security threats identified in the literature include data breaches resulting from weak authentication mechanisms, insecure application programming interfaces, and shared cloud infrastructure. Additionally, ``\textbf{Shadow IT}''---the unauthorized use of cloud applications by employees---has emerged as a significant compounding risk, bypassing established security protocols and increasing the attack surface~\cite{JamesBalancing}. Verma et al.~\cite{Verma2024suggested} highlighted that traditional Single Sign-On (SSO) mechanisms, while convenient, leave SMEs highly vulnerable to phishing and credential theft, necessitating Multi-Factor Authentication (MFA) solutions that avoid the high costs of hardware tokens. Zulkifli et al.~\cite{Zulkifli2023ProposedMFA} similarly proposed a framework utilizing knowledge, possession, and biometric factors to combat single-factor authentication vulnerabilities inherent in SME cloud usage. Network-based attacks such as phishing, botnet activity, session hijacking, eavesdropping, and denial-of-service attacks were reported in studies conducted in India~\cite{Narwane2020Mediating, narwane2019factors}, Turkey~\cite{Sayginer2021Multi}, and Saudi Arabia~\cite{Gibreel2023Factors}. Insider threats, including risks posed by former employees, were also documented, particularly in environments with weak access governance.

The literature further highlights that SMEs face heightened exposure due to limited resources for incident response and recovery. Security controls are often deprioritized in favor of system functionality and cost minimization, creating structural vulnerabilities. The complexity of available security solutions and the requirement for specialized expertise constitute additional barriers, particularly for resource-constrained SMEs lacking dedicated security personnel~\cite{Pruzan2015Monitoring, Tselios2022Improving}.

\subsection{Regulatory Barriers}

Regulatory and legal constraints represent a second major category of adoption barriers. Across the reviewed studies, regulatory challenges primarily manifest through legislative gaps, fragmented compliance requirements, and insufficient government support for SMEs.

In Lebanon, weak infrastructure and the absence of coordinated government initiatives were found to negatively influence cloud adoption decisions~\cite{Skafi2020Factors}. In India, the lack of SME-specific data protection laws creates legal ambiguity regarding liability and compliance responsibilities~\cite{Narwane2020Mediating}. Similarly, studies from Saudi Arabia emphasize this ``trust deficit,'' identifying that widespread ``regulatory ambiguity'' regarding data sovereignty remains a primary barrier despite national digitization mandates~\cite{Alqahtani2024ACCM}. In Bangladesh, Khayer et al.~\cite{Khayer2021AdoptionCC} identified that the lack of a legal framework and regulatory support significantly inhibited cloud adoption, with privacy risks being a major concern. Conversely, in Ghana, the enactment of specific legislation such as the \textit{Data Protection Act} and \textit{Electronic Transaction Act}---coupled with improvements in 4G and fibre-optic infrastructure---has been explicitly cited by SME leaders as a critical enabler that creates a ``safe'' environment for adoption~\cite{Ansong2023Reaching}. This suggests that specific, enforceable legal frameworks may be more effective than broad national vision statements in driving SME confidence.

Cross-border regulatory issues further complicate adoption. The absence of safe-harbor agreements in Turkey restricts market entry by international cloud providers~\cite{Sayginer2021Multi}, while strict data protection and localization requirements in Russia have discouraged investment in local cloud infrastructure. At the same time, SMEs face compliance risks related to global standards such as GDPR, HIPAA, and PCI DSS, which are often designed with large enterprises in mind and impose disproportionate burdens on smaller organizations.

These findings suggest that regulatory environments in developing regions frequently amplify uncertainty rather than mitigate risk, thereby discouraging cloud adoption or leading to informal and insecure deployment practices.

\subsection{Operational Challenges}

Operational barriers constitute a third layer of constraint and are closely intertwined with security and regulatory issues. Migration complexity and system integration challenges were consistently reported as high-impact obstacles. SMEs often lack the technical expertise and project management capacity required to migrate legacy systems to cloud platforms while maintaining business continuity.

Integration difficulties arise from dependence on third-party services, compatibility issues with existing applications, and lack of standardized interfaces. Resource scarcity—both financial and human—was identified as a pervasive constraint, limiting SMEs’ ability to plan, execute, and sustain cloud transitions. Studies report high prevalence of skills shortages, with limited digital expertise and high labor costs exacerbating operational risk.

Organizational resistance to change further constrains adoption. In several studies, lack of top management support and fear of operational disruption were cited as psychological barriers, even when technical feasibility existed. Vendor lock-in concerns also emerged as a moderate but persistent issue, with a significant proportion of SMEs uncertain about how to mitigate dependency on specific cloud providers.

Collectively, these operational challenges impose ongoing burdens related to monitoring service quality, managing costs, and responding to regulatory changes, further complicating secure cloud deployment.

\subsection{Technical Barriers}

Technical infrastructure limitations remain particularly pronounced in developing regions and directly affect both adoption feasibility and security posture. Poor IT infrastructure was identified as a high-impact barrier in countries such as Lebanon~\cite{Skafi2020Factors} and Turkey~\cite{Sayginer2021Multi}, while limited broadband availability and high bandwidth costs were reported as critical constraints in non-metro regions of India~\cite{Wilson2015Enablers}.

Lack of interoperability standards and concerns over vendor lock-in contribute to hesitancy among potential adopters. Compatibility issues and the inherent complexity of cloud applications introduce learning curves that many SMEs struggle to overcome. In Turkey, the need for enhanced fiber infrastructure was identified as essential for supporting cloud services, while integration costs and reliance on high-level IT staff further increased technical barriers~\cite{Sayginer2021Multi}.

In certain contexts, such as the Philippines, inadequate connectivity combined with exposure to natural disasters introduces additional operational and security risks, highlighting the influence of local environmental factors on technical feasibility~\cite{Matias2019Cloud}. Interestingly, while technical hurdles are often cited as primary barriers, some empirical evidence suggests a shifting landscape. Narwane et al.~\cite{narwane2019factors} found that ``technical issues'' were not a statistically significant barrier to adoption in their final structural model for Indian SMEs, suggesting that as cloud platforms mature, the barrier shifts from pure technical feasibility to specific performance metrics---such as the latency of security controls. For example, Verma et al.~\cite{Verma2024suggested} noted that high network latency (200ms ping times) in local SME infrastructure caused email-based OTP codes to be delayed by up to 5 minutes, rendering standard security mechanisms practically unusable without local adaptation.

\section{Local Constraints and Risk Amplification}


Beyond discrete adoption barriers, the reviewed literature demonstrates that local contextual constraints function as risk amplifiers, intensifying the impact of security, regulatory, and operational challenges faced by SMEs in developing regions. These constraints shape not only whether cloud adoption occurs, but also the level of residual risk following deployment.

\subsection{Connectivity Limitations}

Internet infrastructure quality and availability significantly influence cloud adoption feasibility and risk exposure. Broadband availability and high bandwidth costs were identified as primary constraints in India~\cite{Wilson2015Enablers}, while inadequate broadband access and absence of national broadband strategies characterize the Turkish context~\cite{Sayginer2021Multi}. Indonesian SMEs similarly require faster and more reliable connectivity to effectively leverage cloud services~\cite{P2016Point}. In the Philippines, Matias and Hernandez~\cite{Matias2019Cloud} identified that inadequate connectivity, combined with exposure to natural disasters, poses a dual threat that functions as a ``risk amplifier,'' where cloud dependence during a connectivity outage could prove fatal to business operations.

Unreliable connectivity undermines security controls by disrupting authentication processes, monitoring systems, and backup operations. As a result, SMEs often prioritize ease of use and basic functionality over robust security configurations. Poor connectivity thus acts simultaneously as an adoption barrier and a security risk multiplier.

\subsection{Limited Expertise}

Skills gaps and deficiencies in technical knowledge consistently emerge as critical constraints across both developing and developed contexts. In developing regions, SMEs frequently lack the foundational cloud and cybersecurity skills required to configure, monitor, and secure cloud environments. The need for skilled IT staff represents a significant barrier in Jordan~\cite{Al2018Factors} and Indonesia~\cite{P2016Point}, where funding such expertise is often infeasible.

In more mature environments, including parts of Europe, SMEs still demonstrate limited cybersecurity awareness, frequently treating security as an auxiliary concern rather than an integral operational requirement. Training needs were highlighted in Turkey and Bahrain~\cite{Milhem2025integrated}, while studies from Montenegro reported that 43\% of firms lacked sufficient in-house expertise and 38\% found external expertise prohibitively expensive~\cite{Nikolic2025FIT4HPC}.

\subsection{Budget Constraints}

Financial limitations represent the most universal constraint identified across the reviewed literature. While cloud computing reduces upfront capital expenditure, ongoing costs related to infrastructure, security, training, and compliance remain substantial. Economic conditions in Jordan directly influence adoption decisions~\cite{Al2018Factors}, while Lebanese SMEs exhibit low investment in research and development and weak innovation ecosystems~\cite{Skafi2020Factors}.

Budget constraints frequently force SMEs to prioritize cost minimization over security investment, creating a cycle of vulnerability. Although pay-as-you-go cloud models and outsourcing to software-as-a-service providers offer partial mitigation, underinvestment in security controls remains a persistent risk factor.

\subsection{Regional and Cultural Factors}

Regional cultural, economic, and political conditions further shape cloud adoption patterns. Political instability, regulatory uncertainty, and macroeconomic volatility pose significant challenges in Lebanon~\cite{Skafi2020Factors}. In Saudi Arabia, while the government's ``Vision 2030'' initiative actively promotes digital transformation, ``regulatory ambiguity'' remains a critical barrier. Alqahtani~\cite{Alqahtani2024ACCM} identified that despite high-level support, the lack of clear, SME-specific legal frameworks for data protection creates hesitation, with ``Social Context'' factors like culture and awareness playing a statistically significant role in adoption decisions. Similarly, in Qatar, cloud adoption among SMEs remains strikingly low at approximately 3\%, attributed largely to a shortage of qualified strategic leadership and a lack of an ``entrepreneurial society'' that values long-term digital strategy over immediate operational fixes~\cite{Ibrahim2022Antecedents}.
Culture and social capital also influence technology adoption behaviors. In Vietnam, the pandemic highlighted how social factors, including family responsibilities and remote work dynamics, intersect with digital adoption~\cite{Huy2023Big}. In earthquake-prone regions like Aceh, Indonesia, cloud computing is viewed less as an efficiency tool and more as a mechanism for ``\textbf{Digital Resilience},'' enabling SMEs to maintain business continuity and market reach through e-commerce processing during physical disasters~\cite{Chan2023Digital}.
This ``survivalist'' motivation is echoed in other Asian contexts. In the Philippines, Matias and Hernandez~\cite{Matias2019Cloud} found that standard ``Perceived Benefits'' were not statistically significant drivers of adoption; instead, adoption was driven by \textbf{environmental pressure} and regulatory compliance. Similarly, in Indian Tier-II/III cities, Misra et al.~\cite{Misra2022Factors} identified ``\textbf{Perceived Vulnerability}''---specifically the fear of business closure during the pandemic---as a stronger determinant than efficiency gains, prompting the integration of \textbf{Protection Motivation Theory (PMT)} into adoption models. In Russia, economic sanctions introduce unique challenges related to supply chains and infrastructure availability, while in Montenegro, despite a positive business outlook and high potential for cloud HPC adoption, the limited market size remains a structural barrier to the development of specialized services~\cite{Nikolic2025FIT4HPC}.

Together, these factors demonstrate that cloud adoption and security risk among SMEs cannot be fully understood without accounting for local contextual constraints that systematically amplify exposure.

% ----------------------------------------------------------
\section{Existing Frameworks and Solutions for SME Cloud Adoption}
\label{sec:sec25}

The reviewed literature proposes a variety of theoretical models, practical tools, and security frameworks intended to explain or support cloud adoption among SMEs. While these contributions provide valuable insights, they remain fragmented and uneven in their ability to guide secure and sustainable cloud deployment under the constraints faced by SMEs in developing regions.

\subsection{Theoretical Frameworks for Understanding Adoption}

The Technology-Organization-Environment (TOE) framework emerges as the dominant theoretical lens across the reviewed studies. TOE is frequently employed either as a standalone model or in combination with complementary theories such as Diffusion of Innovation (DOI) and the Technology Acceptance Model (TAM). Its appeal lies in its flexible structure, which allows researchers to examine technological readiness, organizational capacity, and environmental influences simultaneously~\cite{Skafi2020Factors, Sayginer2021Multi}.

Empirical applications of TOE across Middle Eastern and Asian SME contexts consistently identify organizational factors---particularly top management support and prior IT experience---as significant predictors of adoption. Environmental factors such as regulatory conditions and competitive pressure also play a notable role, while technological factors alone are often insufficient to explain adoption behavior. For example, studies conducted in Turkey report that top management support and system complexity together explain only 29.8\% of the variance in adoption decisions, indicating substantial unexplained influence from contextual factors~\cite{Sayginer2021Multi}. Asiaei and Rahim~\cite{Asiaei2019Multifaceted} extended this with Diffusion of Innovation (DOI) theory for Malaysian SMEs, identifying relative advantage and compatibility as key drivers, while Fen and Ping~\cite{Fen2024Cloud} highlighted the critical role of cybersecurity readiness within the TOE model for manufacturers.

Several studies extend TOE to address perceived limitations. The \textbf{ACCM-SME} (Adoption of Cloud Computing Model for Saudi SMEs) framework integrates technological, organizational, and environmental contexts with a fourth ``\textbf{Social Context},'' evaluating 17 distinct adoption factors~\cite{Alqahtani2024ACCM}. Validated against data from 412 Saudi SMEs, the model demonstrated that social factors---specifically ``Culture,'' ``Awareness,'' and ``Trust''---significantly enhance the explanatory power of traditional TOE models in the region. Similarly, Ibrahim and Abdullah~\cite{Ibrahim2022Antecedents} proposed integrating TOE with the \textbf{HOT-fit} (Human-Organization-Technology fit) model to better capture the ``human'' element, arguing that lack of qualified personnel and management resistance are as critical as technical compatibility in contexts like Qatar. Combinations of TOE with TAM, UTAUT, and \textbf{Protection Motivation Theory (PMT)} are increasingly used to capture risk-based decision making. For instance, Misra et al.~\cite{Misra2022Factors} successfully integrated UTAUT with PMT to model adoption as a ``coping mechanism'' for SMEs facing widespread business threats, demonstrating that ``threat appraisal'' is as powerful a driver as ``opportunity appraisal'' in crisis contexts~\cite{Milhem2025integrated}.

Overall, the literature demonstrates that while TOE-based models are effective for structuring adoption analysis, they function primarily as explanatory instruments. They do not provide prescriptive guidance for secure deployment, nor do they adequately address the operational and security challenges that arise after adoption.

\subsection{Practical Tools and Deployment-Oriented Platforms}

Beyond theoretical models, several practical tools and platforms are proposed to support SME cloud adoption. CloudRecoMan is frequently cited as a notable example, designed specifically to assist non-technical SME managers in translating business requirements into cloud service recommendations~\cite{Mettouris2022CloudRecoMan}. Its cloud-provider-agnostic design and implementation on an open-source platform enhance accessibility, cost-effectiveness, and maintainability.

Other tools focus on specific aspects of adoption readiness or migration. The \textbf{HPC4SME Assessment Tool}, developed within the EuroCC project, provides an automated self-evaluation aimed at quantifying cloud-HPC readiness~\cite{Nikolic2025FIT4HPC}. It evaluates ``Readiness,'' ``Cloud Potential,'' and ``HPC Performance,'' successfully identifying that while Montenegrin SMEs possess high trust and willingness (Readiness), they often lack the technical workload to justify immediate adoption. The \textbf{Cloud Software Life-Cycle Process (CSLCP)} model, proposed for Egyptian SMEs, provides a structured decision-making framework that prioritizes the acquisition of ready-made SaaS or micro-services over custom development~\cite{Alshazly2020Conceptual}. Validated through an e-commerce case study, CSLCP demonstrates how SMEs can reduce budget risks and delivery times by integrating existing cloud components (e.g., AWS Cognito) rather than building from scratch. Vuong and Braun~\cite{Vuong2015Towards} proposed a secure data storage architecture for multi-tenant CRM solutions, demonstrating that tenant-specific encryption is feasible for SMEs without significant performance degradation. Migration decision support systems developed for SMEs in India emphasize cost-benefit analysis and risk identification~\cite{Narwane2020Mediating}, while platforms such as CloudCxQERP address mobile ERP deployment through dynamic service composition~\cite{Reffad2016Cloud}.

While these tools demonstrate practical utility, they remain narrowly scoped. Most address isolated stages of the adoption lifecycle---such as readiness assessment or service selection---without integrating security governance, operational sustainability, and continuous risk management into a unified deployment framework. However, emerging service models like \textbf{Big Data as a Service (BDaaS)} are beginning to bridge this gap by allowing SMEs to bypass infrastructure heavy-lifting entirely, engaging in advanced business intelligence and analytics through purely cloud-based subscriptions~\cite{Huy2023Big}. Hari et al.~\cite{Hari2022Cloud} demonstrated the reliability of open-source private cloud implementations using OpenStack as a cost-effective alternative for SMEs during the pandemic, achieving 0\% error rates in load tests.

\subsection{Security Frameworks and Guidelines}

For security-specific guidance, the literature frequently references established frameworks such as the NIST Cybersecurity Framework, ISO/IEC 27001, Zero Trust Architecture, and CIS Benchmarks. These frameworks provide structured approaches to risk management, access control, and secure configuration, and are theoretically adaptable to SME contexts.

However, their practical implementation poses challenges for resource-constrained SMEs. ISO/IEC 27001, for example, requires sustained organizational commitment and documentation overhead that many SMEs cannot support. Zero Trust principles offer conceptual clarity but demand architectural maturity that is often absent in small organizations, although recent studies suggest that cloud-native tools are beginning to make Zero Trust affordable and accessible even for smaller firms~\cite{JamesBalancing}. CIS Benchmarks provide free and actionable guidance, yet require technical expertise to apply consistently.

Several studies propose lightweight or modular security approaches tailored to SMEs. A security analytics service designed for Industrial IoT environments emphasizes minimal additional infrastructure by leveraging \textbf{containerized deployments (Docker)} and \textbf{digital twin} models to simulate future attack vectors~\cite{Empl2021Flexible}. Similarly, Verma et al.~\cite{Verma2024suggested} proposed a cost-effective ``hardware-free'' MFA framework specifically for Indian SMEs. By utilizing email-based One-Time Passwords (OTPs) generated via VBScript and replacing standard VPNs with SSL-VPNs, the framework eliminates the capital cost of biometric scanners or hardware tokens while addressing the critical need for layered authentication. Such approaches demonstrate that effective security improvements can be achieved without extensive capital investment, provided that solutions are aligned with SME capabilities.

\subsection{Automation as an Enabling Mechanism}

Automation emerges as a critical enabler of cost-effective cloud security across the reviewed literature. Infrastructure-as-Code tools, such as Terraform and cloud-native provisioning services, are highlighted for enforcing consistent security configurations. Configuration assessment tools enable regular scanning for misconfigurations, while cloud-native logging and monitoring services support incident detection and response with minimal overhead~\cite{Alshazly2020Conceptual}.

Despite their potential, these tools are rarely integrated into comprehensive SME-oriented deployment frameworks. Instead, they are presented as isolated technical options, leaving SMEs without clear guidance on sequencing, governance, or long-term maintainability.

\section{Evaluation and Validation Approaches in the Literature}

The reviewed studies employ a wide range of evaluation methodologies to assess cloud adoption factors and framework effectiveness in SME contexts. However, the prevailing reliance on perceptual and cross-sectional measures limits the ability to assess real-world security and operational outcomes.

Survey-based methods dominate the literature, with studies conducted across the Middle East and Asia using sample sizes ranging from fewer than 20 respondents to over 400 participants. These studies commonly apply Structural Equation Modeling (SEM), confirmatory factor analysis, and regression techniques to validate theoretical relationships~\cite{Misra2022Factors, Gibreel2023Factors}. While such methods provide statistical rigor, they primarily measure intention, perception, and self-reported readiness rather than actual deployment performance.

Case studies and mixed-method approaches offer richer contextual insight but are typically limited to single organizations or short observation periods~\cite{Huy2023Big}. Only one identified study reported objective security outcomes, demonstrating a measurable reduction in security incidents following policy implementation supported by a monitoring tool~\cite{Pruzan2015Monitoring}. This case represents an exception rather than the norm.

Pilot deployments and usability-focused evaluations are rare. The CloudRecoMan platform evaluation emphasized task effectiveness, efficiency, and usability through real-world SME testing, yet did not measure long-term security posture or cost trajectories~\cite{Mettouris2022CloudRecoMan}. Verma et al.~\cite{Verma2024suggested} provided rare quantitative performance metrics from a pilot implementation, measuring the delivery time of email-based OTPs (ranging from 30 seconds to 5 minutes) and network latency. However, even these operational validations are often limited in scope and duration. Cost-efficiency metrics are similarly underdeveloped, with most studies relying on perceived cost savings rather than concrete financial indicators. One notable exception reported projected return on investment expectations among SMEs, though these remained anticipatory rather than realized~\cite{Skafi2020Factors}.

Overall, the literature reveals a significant evaluation gap. Most proposed frameworks and tools are validated through perception-based measures rather than objective indicators such as security incident frequency, configuration compliance, or operational cost evolution. Longitudinal studies examining post-adoption outcomes remain largely absent, limiting confidence in the practical effectiveness of existing approaches.
% ----------------------------------------------------------
\section{Critical Synthesis}
\label{sec:sec26}

The heterogeneity observed across the reviewed literature reflects genuine contextual variation rather than methodological inconsistency. Differences in reported barriers, adoption drivers, and proposed solutions are closely linked to variations in regional infrastructure maturity, organizational capacity, and environmental stability. Consequently, understanding cloud adoption among SMEs in developing regions requires careful interpretation of findings within their specific socio-technical contexts rather than aggregation into uniform conclusions.

\subsection{Contextual Interpretation of Security Concerns}

The stark variation in reported security concerns---ranging from 14.3\% in Jordan~\cite{Al2018Factors} to 56\% in Lebanon~\cite{Skafi2020Factors}---is not merely statistical noise. It reflects a fundamental divergence in risk perception driven by environmental stability. In volatile political or economic climates, SMEs are naturally more risk-averse; they lack the safety nets to absorb a cyber-shock. Conversely, in more stable ecosystems, specific threats may be drowned out by the demand for growth. This implies that security frameworks cannot be static; they must be dynamic enough to calibrate risk controls to the specific ``threat appetite'' of the local environment.

\subsection{Infrastructure Versus Expertise: A Maturity Curve}

The literature hints at a maturity curve that existing frameworks often miss. In early-stage markets like Lebanon, the binding constraint is physical: if the internet cuts out, the cloud is useless~\cite{Skafi2020Factors}. As infrastructure stabilizes, the bottleneck shifts to human capital—the lack of skills to secure the now-accessible cloud~\cite{Narwane2020Mediating}. Recognizing this sequencing is critical. A framework that prescribes advanced identity management to an SME struggling with basic connectivity is fundamentally ignoring the user's reality.

\subsection{Framework Effectiveness and Theoretical Limitations}

The widespread adoption of TOE-based models reflects their flexibility rather than their explanatory completeness. Extensions that incorporate contextual or social factors demonstrate improved explanatory power, yet substantial variance in adoption behavior remains unexplained~\cite{Sayginer2021Multi}. Empirical findings showing the limited predictive strength of perceived benefits challenge the assumption that cloud adoption follows rational, benefit-driven decision-making. Instead, adoption emerges as a risk-management decision constrained by organizational capacity and environmental uncertainty. Current theoretical models explain adoption intention effectively but offer little guidance for secure deployment or post-adoption risk management~\cite{Owusu2020Determinants}.

\subsection{Reframing the Cost--Security Trade-off}

The literature consistently identifies budget constraints as a universal challenge; however, the assumed trade-off between security and cost efficiency is not fully supported by empirical evidence. Case-based findings demonstrate that significant security improvements can be achieved through policy enforcement, configuration management, and automation rather than capital-intensive investments~\cite{Pruzan2015Monitoring}. This indicates that the critical determinant is not the scale of security investment, but the alignment of solution complexity with SME capability. Existing frameworks rarely operationalize this principle, instead emphasizing either comprehensive controls or abstract best practices without regard to implementability.

\subsection{Evaluation Gaps and Evidence Limitations}

A major limitation across the reviewed studies lies in evaluation methodology. The overwhelming reliance on perceptual and cross-sectional measures restricts insight into actual security posture, operational resilience, and cost trajectories following cloud adoption. Objective outcome measures---such as security incident frequency, configuration compliance, or longitudinal cost efficiency---are largely absent. As a result, the practical effectiveness of proposed frameworks and tools remains insufficiently validated. This methodological gap undermines confidence in existing recommendations and limits their transferability to real-world SME deployments.

\subsection{Converging Success Factors and Implications}

Despite contextual variation, several factors emerge consistently as critical to successful cloud adoption among SMEs. Organizational commitment, particularly top management support, plays a decisive role across regions. Regulatory clarity and institutional support significantly influence adoption decisions, while their absence amplifies perceived risk~\cite{Alqahtani2024ACCM}. Prior IT experience reduces uncertainty and facilitates transition, whereas its absence compounds operational and security challenges~\cite{Matias2019Cloud}. Competitive pressure further acts as an external motivator in more dynamic markets.

Taken together, these converging findings indicate that effective SME cloud deployment frameworks must integrate organizational readiness, environmental constraints, and baseline technical capability rather than addressing these dimensions in isolation. Existing models and tools fall short in translating this integrated understanding into deployable, security-aware guidance.

\section{Benchmark of Existing Approaches for SME Cloud Adoption and Security}
\label{sec:benchmark_sme}

To move beyond descriptive synthesis and address the research problem identified in Section \ref{sec:sec26}, a structured benchmark was conducted to comparatively evaluate existing approaches for cloud adoption and security among SMEs in developing regions. The benchmark is grounded in the systematic review of 32 peer-reviewed studies and evaluates representative theoretical frameworks, security standards, tools, and decision-support platforms against a common set of criteria derived directly from the constraints and gaps identified in the literature.

\subsection{Benchmark Criteria}

Based on the findings of the systematic review, six criteria were defined to reflect the practical requirements of SMEs operating under resource, expertise, and infrastructure constraints:

\begin{description}
    \item[Deployment Guidance (Operational ``How'')] -- the extent to which an approach provides concrete, actionable steps for cloud deployment rather than conceptual or explanatory guidance.
    \item[Security Control Coverage] -- the breadth and depth of security controls addressed, including identity, access, network, application, and monitoring mechanisms.
    \item[Cost Awareness] -- explicit consideration of SME budget constraints, including cost minimization strategies, use of free tiers, or reliance on open-source tooling.
    \item[Automation Support] -- the degree to which automation is leveraged to reduce reliance on scarce human expertise and improve consistency.
    \item[SME Contextual Suitability] -- whether the approach is explicitly designed for SMEs or merely adaptable from enterprise-oriented models.
    \item[Empirical Validation with Measurable Outcomes] -- evidence of real-world validation beyond perception-based surveys, including pilot deployments or quantified security or operational outcomes.
\end{description}

These criteria reflect not only adoption considerations but also post-adoption deployment feasibility and sustainability, which are largely underexplored in existing literature.

Table \ref{tab:benchmark} provides a comparative summary of the discussed approaches against key criteria for secure SME cloud deployment.

\begin{table}[htbp]
    \centering
    \caption{Comparative benchmark of existing approaches}
    \label{tab:benchmark}
    \resizebox{\textwidth}{!}{%
    \begin{tabular}{|l|c|c|c|c|c|l|}
        \hline
        \textbf{Approach} & \textbf{Deployment} & \textbf{Security} & \textbf{Cost} & \textbf{Automation} & \textbf{SME Fit} & \textbf{Empirical Validation} \\
        \hline
        TOE / TAM / DOI & \ding{55} & \ding{55} & \ding{55} & \ding{55} & Partial & Survey-based \\
        \hline
        ACCM-SME & \ding{55} & Partial & \ding{55} & \ding{55} & \ding{51} & Survey-based \\
        \hline
        NIST / ISO 27001 & Partial & \ding{51}\ding{51} & \ding{55} & Partial & Adaptable & Compliance-based \\
        \hline
        CIS Benchmarks & Partial & \ding{51} & \ding{51} & \ding{55} & Adaptable & Configuration \\
        \hline
        CloudRecoMan & \ding{51} & \ding{55} & \ding{51} & Partial & \ding{51}\ding{51} & Pilot \\
        \hline
        CMDSS & \ding{51} & \ding{55} & Partial & \ding{55} & \ding{51} & Conceptual \\
        \hline
        SME-MFA (Verma/Zulkifli) & Partial & \ding{51} & \ding{51} & \ding{51} & \ding{51} & Pilot / Framework \\
        \hline
        Secure CRM Arch. & \ding{51} & \ding{51} & Partial & Partial & \ding{51} & Prototype \\
        \hline
        Private Cloud (OpenStack) & \ding{51} & \ding{51} & \ding{51}\ding{51} & \ding{51} & Adaptable & Performance Testing \\
        \hline
        SkyHigh & \ding{55} & \ding{51} & \ding{55} & \ding{51} & Partial & Quantified case \\
        \hline
    \end{tabular}%
    }
\end{table}

\subsection{Benchmark Findings}

\subsubsection*{Deployment Guidance}
The benchmark reveals that most theoretical adoption frameworks, including TOE-, TAM-, and DOI-based models, provide valuable explanatory insight into adoption intentions but offer no operational guidance for secure cloud deployment. Similarly, SME-specific adoption models such as ACCM-SME focus on identifying influential factors rather than prescribing deployment steps.

In contrast, practical platforms such as CloudRecoMan and CMDSS provide partial deployment guidance by translating business requirements into recommended cloud configurations or migration paths. Targeted architectural solutions, such as the Secure CRM Architecture proposed by Vuong and Braun~\cite{Vuong2015Towards}, offer specific implementation blueprints for multi-tenant environments. Infrastructure-as-Code tools further strengthen deployment guidance by enabling reproducible and policy-driven provisioning; however, these tools are rarely integrated into a broader SME-oriented framework.

\subsubsection*{Security Control Coverage}
Security frameworks such as NIST and ISO/IEC 27001 offer comprehensive coverage of security domains and remain the most complete references for risk management. However, they are primarily control-oriented and assume organizational maturity that many SMEs lack.

Other approaches, including Zero Trust principles, CIS Benchmarks, and specialized MFA frameworks (e.g., Verma et al.~\cite{Verma2024suggested}, Zulkifli et al.~\cite{Zulkifli2023ProposedMFA}), address specific security layers effectively but do so in isolation. Monitoring tools such as SkyHigh provide visibility and incident detection yet do not address preventive controls or deployment sequencing. Overall, security coverage in the literature is fragmented across tools and frameworks rather than integrated.

\subsubsection*{Cost Awareness}
Cost emerges as the most influential adoption factor across the reviewed studies, yet explicit cost modeling or optimization guidance is largely absent from most frameworks. While some tools and approaches implicitly address cost through the use of open-source platforms, free tiers, or avoidance of specialized hardware, these strategies are not systematized.

Exceptions include the private cloud implementation study by Hari et al.~\cite{Hari2022Cloud}, which demonstrated the viability of OpenStack to achieve enterprise-grade reliability (0\% error rates) without licensing costs. Similarly, lightweight MFA frameworks explicitly design "hardware-free" solutions to minimize capital expenditure. CloudRecoMan also integrates cost as a primary selection criterion. In contrast, enterprise-oriented security standards largely ignore cost considerations, reinforcing the cost--security trade-off faced by SMEs.

\subsubsection*{Automation Support}
Automation is consistently identified in the literature as a critical enabler for secure cloud deployment under limited expertise. Infrastructure automation tools, security scanning utilities, and automated assessment platforms demonstrate strong potential to reduce configuration errors and operational burden.

Despite this, automation is typically treated as a technical option rather than a core design principle. Most adoption frameworks and lifecycle models do not explicitly incorporate automation into their structure, leaving SMEs without guidance on how automation should be systematically applied across deployment and security operations.

\subsubsection*{SME Contextual Suitability}
Several approaches explicitly target SMEs or specific regional contexts, improving relevance and applicability. However, even SME-focused models often emphasize adoption decision-making rather than operational execution. Widely used security frameworks are described as ``adaptable'' to SMEs but were not originally designed with SME constraints in mind, resulting in scalability and complexity challenges.

The benchmark highlights a recurring mismatch between the theoretical adaptability of frameworks and their practical usability by SMEs lacking dedicated security personnel or mature governance structures.

\subsubsection*{Empirical Validation}
The most significant weakness identified through the benchmark is the lack of empirical validation with measurable outcomes. The majority of studies rely on survey-based methods and structural modeling to validate adoption factors rather than evaluating deployed systems.

Only a small number of studies report pilot implementations or quantified outcomes, such as the registered performance metrics of OpenStack deployments~\cite{Hari2022Cloud}, latency measurements of MFA prototypes~\cite{Verma2024suggested}, or the documented reduction in security incidents from monitoring solutions~\cite{Pruzan2015Monitoring}. Longitudinal evaluations and cost--security performance measurements remain largely absent.

\begin{figure}[htbp]
    \centering
    \includegraphics[width=\textwidth]{Figures/SME_cloud_adoption_research_framework_diagram.png}
    \caption{SME Cloud Adoption Research Framework}
    \label{fig:framework_diagram}
\end{figure}

\subsection{Synthesis and Implications}
\label{sec:benchmark_synthesis}

The benchmark demonstrates that while existing approaches address individual aspects of SME cloud adoption and security, none provide integrated coverage across all six criteria. Theoretical models explain adoption behavior but lack operational guidance. Security frameworks offer comprehensive controls but impose complexity and cost burdens. Practical tools and platforms address specific gaps yet remain fragmented and insufficiently validated.

Most critically, the benchmark reveals the absence of a fully integrated, empirically validated framework that combines deployment guidance, security controls, cost awareness, automation, and SME contextualization into a coherent and practical roadmap. This absence is not due to a lack of tools or standards, but to the lack of integration and prioritization tailored to SME realities.

These findings directly justify the need for the Secure Cloud Deployment Framework (SCDF) proposed in this study, which aims to bridge the gap between high-level security principles and deployable, cost-aware, and automated cloud security practices suitable for SMEs in developing regions.

% ----------------------------------------------------------
\section*{Summary}
\addcontentsline{toc}{section}{Summary}

This chapter critically reviewed the existing literature on cloud computing adoption among small and medium-sized enterprises (SMEs), with particular emphasis on cybersecurity challenges, contextual constraints, and the applicability of existing frameworks in developing regions. The review demonstrated that while prior research provides substantial insight into the factors influencing cloud adoption decisions, it largely concentrates on adoption intention rather than secure and sustainable deployment.

The findings indicate that SMEs face a combination of security, regulatory, operational, and technical barriers, which are further amplified by local constraints such as limited connectivity, skills shortages, and budget restrictions. Although several theoretical models and practical tools have been proposed, most existing frameworks are either overly complex, enterprise-oriented, or insufficiently adapted to the operational realities of SMEs.

Furthermore, the literature reveals a significant evaluation gap, with the majority of studies relying on perceptual measures rather than objective security and operational outcomes. As a result, there remains limited guidance on how SMEs can deploy cloud services securely, cost-effectively, and sustainably under real-world constraints.

These findings expose a critical gap. We know why SMEs want the cloud, and we know why they struggle to secure it. What is missing is the bridge between the two: a practical, ``how-to'' deployment framework that acknowledges the messy reality of limited budgets, scarce skills, and unstable infrastructure. The existing literature offers high-level theory or rigid enterprise standards; it does not offer a roadmap for the small business in a developing region. This study aims to build that roadmap.

The next chapter outlines the research methodology designed to construct and validate this proposed Secure Cloud Deployment Framework (SCDF).

