\chapter{Literature Review}
\label{chap:second}

\ifpdf
    \graphicspath{{Chapter2/Figures/PNG/}{Chapter2/Figures/PDF/}{Chapter2/Figures/}}
\else
    \graphicspath{{Chapter2/Figures/EPS/}{Chapter2/Figures/}}
\fi

\section{Introduction}

\begin{itemize}
    \item \textbf{Cloud Adoption:} Cornerstone of SME digital transformation (cost/scale/flexibility).
    \item \textbf{Influencing Factors:} Organizational readiness, environmental pressure, risk perception.
    \item \textbf{Developing Regions:} Structural constraints (internet, regulation, skills) + high threat exposure.
    \item \textbf{Cybersecurity:} Critical but unevenly addressed. Models often identify risks without offering solutions.
    \item \textbf{Frameworks:} Existing models (NIST, etc.) designed for large enterprises, creating a gap for SMEs.
    \item \textbf{Chapter Goal:} Review literature on adoption, challenges, and frameworks to justify the need for SCDF.
\end{itemize}

% ----------------------------------------------------------
\section{Cloud Computing Adoption among SMEs}
\label{sec:sec21}

\begin{itemize}
    \item \textbf{Adoption Drivers:} Competitiveness, Agility, OpEx preference over CapEx.
    \item \textbf{Literature Consensus:} SMEs adopt for economic/operational benefits (Cost, Scalability).
\end{itemize}

\subsection{Determinants of Cloud Adoption in SMEs}

\begin{itemize}
    \item \textbf{Frameworks:} TOE (Technology-Organization-Environment), DOI, TAM.
    \item \textbf{Key Factors:} Relative advantage, complexity, management support, competitive pressure.
    \item \textbf{limitation:} Predicts *intention* (yes/no) but ignores *process* (how/securely).
\end{itemize}

\subsection{Adoption Patterns in Developing Regions}

\begin{itemize}
    \item \textbf{Constraints:} Infrastructure (Connectivity/Power), Economic uncertainty.
    \item \textbf{Trend:} Incremental adoption (SaaS favored).
    \item \textbf{Risk:} Dependency on providers + fear of lock-in.
\end{itemize}

\subsection{Limitations of Adoption-Centric Models}

\begin{itemize}
    \item \textbf{Blind Spot 1:} Assumes benefits follow adoption automatically (ignores integration/security effort).
    \item \textbf{Blind Spot 2:} Treats security as abstract "risk perception" variable.
    \item \textbf{Methodology:} Survey dominance misses context-specific implementation realities (informal practices).
\end{itemize}

\subsection{Implications for Secure Cloud Deployment}

\begin{itemize}
    \item Security is an operational requirement, not just an adoption barrier.
    \item Need to view adoption as a lifecycle requiring governance.
\end{itemize}

% ----------------------------------------------------------
\section{Cybersecurity Challenges Facing SMEs}
\label{sec:sec22}

\begin{itemize}
    \item \textbf{Mismatch:} SMEs rely on digital but lack defense capacity (High exposure vs. Low defense).
    \item \textbf{Nature of Challenge:} Detailed by organizational constraints, not just technical ones.
\end{itemize}

\subsection{Resource and Capability Constraints}

\begin{itemize}
    \item \textbf{Personnel:} No CISO/Security team. Reliance on general IT staff or external support.
    \item \textbf{Practice:} Reactive security (fix it when it breaks).
    \item \textbf{Budget:} Immediate survival > Long-term risk mitigation.
\end{itemize}

\subsection{Threat Exposure in Cloud-Enabled Environments}

\begin{itemize}
    \item \textbf{Shared Responsibility:} Often misunderstood. Users fail to secure their side (data/access).
    \item \textbf{Vectors:} Phishing, Ransomware, Credential theft.
    \item \textbf{Visibility:} Lack of logging/monitoring = delayed breach detection.
\end{itemize}

\subsection{Organizational and Human Factors}

\begin{itemize}
    \item \textbf{Culture:} Informal procedures, trust-based access, low security awareness.
    \item \textbf{Decision Making:} Centralized (Owner-driven). Security viewed as "Tech problem", not strategy.
\end{itemize}

\subsection{Challenges in Developing Regions}

\begin{itemize}
    \item \textbf{External Env:} Unreliable internet, incomplete regulation, lack of local security firms.
    \item \textbf{Support:} Reliance on informal networks/low-cost vendors.
\end{itemize}

\subsection{Implications for Secure Cloud Adoption}

\begin{itemize}
    \item Security is a persistent operational risk.
    \item Need: Scalable, simplified guidance focusing on "must-dos" rather than "nice-to-haves".
\end{itemize}
% ----------------------------------------------------------
\section{Existing Cloud Security Frameworks and Standards}
\label{sec:sec23}
\begin{itemize}
    \item \textbf{General:} Frameworks exist (Risk management, Compliance, Protection).
    \item \textbf{Issue:} Built for Enterprise -> Misaligned with SME capacity.
\end{itemize}

\subsection{Overview of Major Cloud Security Frameworks}

\begin{itemize}
    \item \textbf{NIST Cybersecurity Framework:} Identify, Protect, Detect, Respond, Recover. (Requires formal risk assessment).
    \item \textbf{ISO/IEC 27001:} Formal ISMS. (High admin/cost burden for certification).
    \item \textbf{CSA CCM:} Cloud Controls Matrix. (Granularity overwhelms non-experts).
\end{itemize}

\subsection{Framework Assumptions and SME Misalignment}

\begin{itemize}
    \item \textbf{Assumptions:} Dedicated team, formal documentation, segregation of duties.
    \item \textbf{SME Reality:} Informal processes, multitasking staff.
    \item \textbf{Result:} "Implementation Paralysis" or fragmented adoption.
\end{itemize}

\subsection{Cloud Shared Responsibility and Framework Gaps}

\begin{itemize}
    \item \textbf{Gap:} Frameworks don't always operationalize the shared responsibility model.
    \item \textbf{Risks:} SMEs assume "inherited security" and miss their obligations (e.g., config, access rights).
\end{itemize}

\subsection{Towards Context-Aware Security Guidance}

\begin{itemize}
    \item Enterprise standards must be reinterpreted, not just replicated.
    \item Requirement: Prioritize essential controls + Align with SME reality (Simple, Actionable).
\end{itemize}
% ----------------------------------------------------------
\section{Critical Assessment and Research Gap}
\label{sec:sec24}

\begin{itemize}
    \item \textbf{Review Summary:} Progress in understanding "Why Adopt" (Intent), but gap in "How to Secure" (Operation).
    \item \textbf{Gap 1 (Adoption):} Focus on intention/determinants. Security viewed as "barrier" not "process".
    \item \textbf{Gap 2 (Cyber):} Focus on describing vulnerabilities. Lacks actionable solutions/guidance.
    \item \textbf{Gap 3 (Frameworks):} Enterprise-centric. Too heavy/expensive/complex for SMEs.
    \item \textbf{Methodology Gap:} Dominance of quant surveys -> lacks deep contextual insight.
    \item \textbf{Identified Gap:} Absence of integrated, security-focused cloud deployment framework bridging theory and SME constraints.
    \item \textbf{Research Focus:} Develop SCDF to guide SMEs through secure adoption (prioritized, feasible steps).
\end{itemize}

% ----------------------------------------------------------
\section*{Summary}
\addcontentsline{toc}{section}{Summary}

\begin{itemize}
    \item Chapter reviewed literature: SME Cloud Adoption, Cyber Challenges, Frameworks.
    \item Finding: Research explains *adoption* well, but lacks practical *security deployment* guidance.
    \item Challenge: SMEs face resource/skill constraints + high threat.
    \item Current Solutions: Frameworks are mismatched (too complex).
    \item Conclusion: Clear research gap -> Need for SCDF tailored to SME reality in developing regions.
\end{itemize}

