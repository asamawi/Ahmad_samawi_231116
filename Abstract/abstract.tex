\begin{abstracts}

Small and medium-sized enterprises (SMEs) in developing regions increasingly adopt cloud computing to reduce costs and improve operational efficiency. However, limited technical expertise, constrained budgets, and weak local infrastructure often prevent these organizations from deploying cloud environments securely and sustainably. While existing international standards and security frameworks provide comprehensive guidance, they are frequently difficult for SMEs to implement due to their complexity, cost, and skill requirements.

This thesis proposes a Secure Cloud Deployment Framework (SCDF) that translates essential cloud security principles into a practical, step-by-step deployment approach tailored to SMEs in developing regions. The framework prioritizes affordability, incremental adoption, and operational feasibility by mapping security controls to widely available open-source tools and clearly defined implementation stages. Rather than focusing solely on adoption decisions, the framework addresses the post-adoption challenge of secure deployment, guiding SMEs from initial readiness assessment to baseline security hardening.

The study synthesizes existing research, analyzes region-specific constraints, and evaluates the proposed framework through controlled simulations and applied case studies. The expected outcome is a deployable, cost-effective model that enables SMEs to improve cloud security posture with limited resources, supporting resilient digital transformation without reliance on complex or proprietary solutions.

\end{abstracts}
