\begin{abstracts}
\begin{singlespace}
Small and medium-sized enterprises (SMEs) in developing regions seem to be turning to cloud computing in droves, largely to cut costs and tighten up operations. Yet, it’s not always a smooth transition. Limited technical know-how, tight budgets, and shaky local infrastructure often stand in the way of deploying these environments securely—or sustainably. And while there’s no shortage of international standards or security frameworks, they can feel a bit out of reach for smaller players. They tend to be complex, expensive, and demand a level of expertise that many of these businesses just don't have.

This thesis proposes what looks like a more practical path: a Secure Cloud Deployment Framework (SCDF). The idea is to translate high-level cloud security principles into a step-by-step approach that actually works for SMEs in developing regions. Instead of just focusing on the decision to adopt the cloud, this framework tackles what comes next—getting it up and running securely. It favors affordability and taking things one step at a time, mapping security controls to open-source tools that are easy to get hold of.

We’ve tried to ground this work in reality. The study pulls together existing research, takes a hard look at the constraints specific to these regions, and stress-tests the framework through simulations and case studies. The hope is to offer a model that’s not just theoretically sound, but deployable. Ideally, it should empower SMEs to toughen up their security posture without breaking the bank or relying on overly complex, proprietary solutions.
\end{singlespace}
\end{abstracts}
