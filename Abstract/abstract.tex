\begin{abstracts}
\begin{singlespace}
Cloud computing is often presented as a practical way for small and medium enterprises (SMEs) to obtain flexible and affordable computing resources. In reality, adoption remains uneven, particularly in developing regions, where security concerns, limited technical skills, regulatory uncertainty, and financial constraints continue to shape decision-making. Many existing cloud security standards and best-practice frameworks were developed for large organizations and assume levels of expertise and resources that are rarely available within SMEs.

This research introduces a Secure Cloud Deployment Framework (SCDF) intended to address this mismatch. The framework draws selectively on established security standards, including NIST and ISO guidance, but adapts them into a step-by-step and risk-oriented deployment approach that is more aligned with SME capabilities. Emphasis is placed on prioritizing essential security controls, using automation where possible, and relying on open-source tools to limit cost and operational overhead.

The proposed framework is examined through simulations and illustrative SME case studies. Evaluation focuses on practical indicators such as changes in security posture, cost implications, and ease of use rather than on adoption intentions alone. The results indicate that a lightweight and incremental security approach can meaningfully improve cloud security for SMEs without requiring full compliance with complex enterprise-level standards, supporting more sustainable digital transformation in resource-constrained contexts.
\end{singlespace}
\end{abstracts}
