\chapter{Conclusions and Future Work}
\label{chap:conclusions}

\ifpdf
    \graphicspath{{Chapter5/Figures/PNG/}{Chapter5/Figures/PDF/}{Chapter5/Figures/}}
\else
    \graphicspath{{Chapter5/Figures/EPS/}{Chapter5/Figures/}}
\fi

\section*{Summary}
\addcontentsline{toc}{section}{Summary}

This chapter summarizes the research findings, discusses the contributions and limitations of the study, and presents recommendations for future work. The study aimed to design and validate a practical, low-cost Secure Cloud Deployment Framework (SCDF) for small and medium-sized enterprises (SMEs) in developing regions.

% ----------------------------------------------------------
\section{Summary of Findings}
\label{sec:sec51}

This research addressed the growing need for accessible cloud-security solutions among SMEs that lack the resources to implement complex enterprise frameworks.  
The key findings are as follows:

\begin{itemize}
    \item \textbf{Contextual Challenges:} SMEs in developing regions face persistent barriers to cloud adoption, including limited expertise, inadequate infrastructure, and unclear regulatory enforcement \citep{Sabbah2019_LebanonTOE,AlRababah2023_MENA_Barriers}.
    \item \textbf{Framework Design:} The proposed SCDF integrates governance, technical, and operational layers, each adaptable to SME needs. Its modular structure allows gradual implementation aligned with available resources.
    \item \textbf{Practical Feasibility:} The simulation-based validation demonstrated that open-source tools such as Wazuh, Suricata, and the ELK Stack can deliver effective monitoring and threat detection with minimal performance overhead.
    \item \textbf{Affordability and Usability:} The total deployment cost remained under USD~500, confirming that the SCDF is financially viable for SMEs while maintaining strong detection accuracy and compliance coverage.
    \item \textbf{Expert Validation:} Feedback from cybersecurity professionals confirmed that the framework provides a structured, implementable model for organizations operating under technical and budgetary constraints.
\end{itemize}

Collectively, these findings confirm that the SCDF offers a feasible and scalable approach to improving cloud security for SMEs in developing regions.

% ----------------------------------------------------------
\section{Research Contributions}
\label{sec:sec52}

The main contributions of this research can be summarized as follows:

\begin{enumerate}
    \item Development of a \textbf{Secure Cloud Deployment Framework (SCDF)} specifically tailored to the needs of SMEs in developing economies.
    \item Integration of \textbf{international best practices} (NIST, ISO/IEC, ENISA) with \textbf{regionally relevant controls}, ensuring compatibility with local regulations such as Lebanon’s Law No.~81 (2018) \citep{LebanonLaw81_2018_SMEX}.
    \item Demonstration of how \textbf{open-source technologies} can bridge the affordability and accessibility gap in cloud security for SMEs.
    \item Empirical validation of the framework through \textbf{simulation testing and expert evaluation}, contributing practical evidence to the academic and professional domains.
\end{enumerate}

This combination of technical design, regulatory alignment, and empirical validation represents a meaningful step toward enabling secure, sustainable digital transformation for SMEs in developing regions.

% ----------------------------------------------------------
\section{Limitations of the Study}
\label{sec:sec53}

While the results are promising, several limitations must be acknowledged:

\begin{itemize}
    \item The framework was validated using a simulated SME environment; results may vary in production deployments with larger workloads or multi-cloud infrastructures.
    \item Expert evaluations, while informative, were limited in number and geographic scope. Broader participation from different countries and industries could further validate generalizability.
    \item The study focused primarily on technical and governance aspects; economic and behavioral factors influencing SME security adoption were beyond its scope.
    \item The evaluation timeframe was relatively short. Long-term monitoring and incident response data would strengthen confidence in framework resilience.
\end{itemize}

Recognizing these limitations provides valuable direction for future research and implementation refinement.

% ----------------------------------------------------------
\section{Recommendations for Practice}
\label{sec:sec54}

Based on the findings, the following practical recommendations are proposed:

\begin{enumerate}
    \item \textbf{For SMEs:} Adopt a phased approach to implementing SCDF, starting with identity management, basic monitoring, and data backup before scaling to advanced intrusion detection and automation.
    \item \textbf{For Policymakers:} Support the development of localized cybersecurity guidelines derived from SCDF principles to enhance national SME readiness.
    \item \textbf{For Industry Associations:} Facilitate training programs and cooperative security initiatives to improve SME awareness and reduce individual costs.
    \item \textbf{For Cloud Providers:} Offer simplified, low-cost compliance templates and open APIs that align with SCDF recommendations for secure multi-tenant deployments.
\end{enumerate}

These recommendations promote the practical adoption of the framework and foster an ecosystem of shared responsibility across stakeholders.

% ----------------------------------------------------------
\section{Future Work}
\label{sec:sec55}

Several opportunities exist to extend this research:

\begin{itemize}
    \item \textbf{Empirical Deployment:} Deploy the SCDF in real SME environments across multiple countries to collect longitudinal data on effectiveness and cost impact.
    \item \textbf{Automation and AI:} Integrate lightweight AI-driven analytics for anomaly detection and automated incident prioritization.
    \item \textbf{Economic Analysis:} Conduct a cost–benefit study comparing SCDF adoption with managed cloud-security services.
    \item \textbf{Policy Integration:} Collaborate with national regulators to adapt SCDF into official SME cybersecurity frameworks, improving standardization and compliance.
\end{itemize}

Future research should continue bridging the gap between academic frameworks and real-world implementation, ensuring that SMEs in developing regions can securely harness the benefits of cloud computing.

% ----------------------------------------------------------
\section*{Summary}
\addcontentsline{toc}{section}{Summary}

This final chapter summarized the findings and contributions of the research, identified its limitations, and outlined recommendations for practice and future work. The proposed Secure Cloud Deployment Framework (SCDF) provides SMEs in developing regions with a structured, scalable, and affordable pathway toward secure cloud adoption—addressing one of the most pressing challenges in modern digital transformation.

